\begin{deluxetable*}{lccccccc}
\tablecolumns{7}
\tablewidth{0pc}
\tablecaption{Flux upper limits for regions of interest in the Coma galaxy cluster}
\tablehead{
\colhead{Source} &
\colhead{RA} &
\colhead{Dec} &
\colhead{RoI} &
\colhead{Excess} &
\colhead{Significance\tablenotemark{a}} &
\multicolumn{2}{c}{Flux U.L.\tablenotemark{b}}\\
\colhead{} & \colhead{[J2000]} & \colhead{[J2000]} & \colhead{[deg]} & \colhead{[counts]} & \colhead{[$\sigma$]} & \multicolumn{2}{c}{[$10^{-8}$ ph. m$^{-2}$ s$^{-1}$]}}
\startdata
Core & \RA{12}{59}{48.7} & Dec{27}{58}{50.0} & 0 & 17 & 0.84 & 2.62 & (1.3\%)\\
& & & 0.2 & -41 & -1.0 & 1.7 & (0.87\%)\\
& & & 0.4 & -26 & -0.30 & 4.5 & (2.3\%)\\
NGC 4889 & \RA{13}{00}{08.13} & \Dec{+27}{58}{37.03} & 0 & 3 & 0.14 & 2.09 & (1.1\%)\\
NGC 4874 & \RA{12}{59}{35.71} & \Dec{+27}{57}{33.37} & 0 & -14 & -0.71 & 1.42 & (0.72\%)\\
NGC 4921 & \RA{13}{01}{26.12} & \Dec{+27}{53}{09.59} & 0 & -4 & -0.23 & 1.89 & (0.96\%)
\enddata
\tablenotetext{a}{Statistical significance calculated according to \citet{article:LiMa:1983}}
\tablenotetext{b}{99\% confidence level upper limit calculated according to \citet{article:Rolke:2005} above an energy threshold of 220 GeV. Values in parentheses are the corresponding fluxes in percent of the steady Crab nebula flux.}
\label{table:results}
\end{deluxetable*}
