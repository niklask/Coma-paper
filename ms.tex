\documentclass[12pt,manuscript]{aastex}
%\documentclass[twocolumn]{emulateapj}

\def\del#1{{}}
%\def\del#1{{\bf (DELETED TEXT)}}
\def\C#1{#1}
%\def\C#1{{\bf #1}}

\newcommand{\vdag}{(v)^\dagger}
\newcommand{\myemail}{nkarlsson@physics.umn.edu}
\newcommand{\expval}[1]{\left\langle #1 \right\rangle}
\newcommand{\RA}[3]{#1$^{\mathrm{h}}$#2$^{\mathrm{m}}$#3$^{\mathrm{s}}$}
\newcommand{\Dec}[3]{#1$^{\circ}$#2\arcmin#3\arcsec}
\newcommand{\rmn}{\mathrm}
\newcommand{\CR}{\mathrm{CR}}
\newcommand{\dps}{\displaystyle}
\newcommand{\eps}{\varepsilon}
\newcommand{\bra}{\langle}
\newcommand{\ket}{\rangle}
\shorttitle{Coma Cluster Observations with VERITAS}
\shortauthors{VERITAS collaboration}

\begin{document}

%\title{VHE Gamma-Ray Observations of the Coma Cluster of Galaxies with VERITAS}
\title{Constraints on Cosmic-Ray Particle Populations, Magnetic Fields, and Dark Matter from VHE Gamma-Ray Observations of the Coma Cluster of Galaxies with VERITAS}
\author{N. Karlsson\altaffilmark{1,2}, M. Pohl\altaffilmark{3,4}}
\affil{(the VERITAS collaboration)} 
%%\email{nkarlsson@physics.umn.edu}
\and
\author{S. Federici\altaffilmark{3,4}, C. Pfrommer\altaffilmark{1,5},
A. Pinzke\altaffilmark{6}}

\altaffiltext{1}{Corresponding authors: N. Karlsson, nkarlsson@physics.umn.edu \& C. Pfrommer, christoph.pfrommer@h-its.org}
\altaffiltext{2}{School of Physics and Astronomy, University of Minnesota, Minneapolis, MN 55455, USA}
\altaffiltext{3}{DESY, Platanenallee 6, 15738 Zeuthen, Germany}
\altaffiltext{4}{Institut f\"ur Physik und Astronomie, Universit\"at Potsdam, 14476 Potsdam-Golm, Germany}
\altaffiltext{5}{Heidelberg Institute	for Theoretical	Studies, Schloss-Wolfsbrunnenweg 35, D-69118 Heidelberg, Germany}
\altaffiltext{6}{Department of Physics, University of California, Santa Barbara, CA 93106, USA}

%%\journalinfo{The Astrophysical Journal}
%%\submitted{preprint}

\begin{abstract}
Theoretical models of galaxy clusters suggest they may be sources of very high-energy (VHE)
gamma-ray emission. Observations of radio halos and relics in clusters indicate populations of
relativistic electrons, which are either injected into the intra-cluster medium and accelerated by
turbulent shocks or secondaries from interactions of cosmic-ray hadrons (protons and ions) with
the intra-cluster gas. Those interactions could also produce gamma rays through the decay of
neutral pions. Reported here are observations of the Coma cluster of galaxies with the VERITAS
array of imaging Cherenkov telescopes. The observations were made between March and May
2008. Radio observations at 1.41 GHz taken with the Green Bank Telescope and X-ray observations
with the ROSAT All-Sky survey was used to a priori select regions of interest in the field of view.
No significant VHE gamma-ray emission from the Coma cluster was detected by VERITAS. Several
emission regions are considered, including the cluster core as both unresolved (point-like) and
spatially extended, and a select few cluster members. The flux upper limits at the 99\% confidence
level measured are on the order of $2-5\times 10^{-8}$ ph. m$^{-2}$ s$^{-1}$, which is two orders
of magnitude higher than model-predicted minimum gamma-ray fluxes.

The gamma-ray flux upper limits have been used to constrain the populations of cosmic-ray particles
and magnetic fields and in the Coma cluster of galaxies. The relative cosmic-ray pressure,
$X_{\mathrm{CR}}=\expval{P_{\CR}}/\expval{P_{\rmn{th}}}$, is on the order of 10-20\% when assuming
a gamma-ray photon index $\alpha=2.3$. The observations imply a lower limit on the intra-cluster
magnetic field on the order of a $\mu$G, depending on the magnetic decline and gamma-ray spectral
index. Finally, a constraint on the thermally averaged product of the total self-annihilation
cross section and the relative velocity of the dark-matter particles, $\expval{\sigma v}$,
is derived and found to be $\lesssim 10^{-21}$ cm$^{3}$ s$^{-1}$ for $m_{\chi}\gtrsim 1000$
GeV/c$^{2}$.
\end{abstract}

%\keywords{cosmic rays: interactions --- galaxies --- galaxy clusters -- VHE}
\keywords{cosmic rays --- gamma rays: observations --- gamma rays: VHE --- galaxies: clusters: general --- galaxies: clusters: individual: Coma (ACO 1656) --- dark matter --- magnetic fields}


%
%

\section{Introduction}
Clusters of galaxies are the largest virialized objects in the Universe, with typical diameters of
a few Mpc and masses on the order of $10^{14}$ to $10^{15} M_{\odot}$. According to the currently
favored hierarchical model of cosmic structure formation, larger objects formed through successive
mergers of smaller objects with galaxy clusters sitting on top of this mass hierarchy
\citep[see][for a review]{article:Voit:2005}. Most of the mass of a cluster is in form of dark
matter (DM), as indicated by galaxy dynamics and gravitational lensing
\citep{article:DiaferioSchindlerDolag:2008}. Baryonic gas making up the intra-cluster medium (ICM)
contributes about 15\% of the total cluster mass and individual galaxies account for the remainder
(about 5\%). The ICM gas mass also comprises a significant fraction of the observable (baryonic)
matter in the Universe.

The ICM is a hot, $T\sim 10^{8}$ K, plasma emitting thermal bremsstrahlung in the soft
X-rays regime \citep[see, e.g.,][]{article:Petrosian:2001}. This plasma has been heated
primarily through collisionless structure-formation shocks that form as a result of the
hierarchical merging and accretion processes. Such shocks and turbulence in the ICM gas and
intra-cluster magnetic fields also provide a means to efficiently accelerate particles \citep[see,
e.g.,][]{article:ColafrancescoBlasi:1998, article:Ryu_etal:2003}. Many clusters do indeed feature
mega-parsec scale halos of non-thermal radio emission, indicative of a population of relativistic
electrons and positrons (leptons) and magnetic fields permeating the ICM
\citep{article:Cassano_etal:2010}. The two theories competing to explain the radio halo
are the ``hadronic model'', in which the radio-emitting electrons are produced in inelastic
collisions of cosmic-ray (CR)  ions with thermal gas of the ICM \citep{article:Dennison:1980},
%\citep[][and references therein]{article:EnsslinPfrommerMiniatiSubramanian:2011}
and the ``re-acceleration model'', in which during states of strong ICM turbulence, e.g. after a
cluster merger, interactions of a long-lived pool of 100-MeV electrons with plasma waves might
be efficient enough to accelerate them to energies ($\sim 10$ GeV) sufficient to produce observable
radio emisson \citep[][and references therein]{article:BrunettiLazarian:2010}. Recent observations of
possibly non-thermal emission from clusters in the the extreme ultraviolet
\citep[EUV; ][]{article:SarazinLieu:1998} and hard X-rays \citep{article:RephaeliGruber:2002, article:Fusco-Femiano_etal:2004, article:Eckert_etal:2007} may provide further indications of
relativistic particle populations in clusters, although the interpretation of these observations as non-thermal diffuse emission have been disputed on the basis of more sensitive observations
\citep[see, e.g.,][]{article:Ajello_etal:2009, article:Ajello_etal:2010, article:Wik_etal:2009}.

Galaxy clusters have for many years been proposed as sources of gamma rays. If shock acceleration
in the ICM is an efficient process, a population of highly relativistic CR protons and heavy ions is to
be expected in the ICM. The main energy-loss mechanism for CR hadrons at high energies is pion
production through the interaction of CRs with nuclei in the ICM. Pions are short lived and decay; the
charged pions to electrons and positrons,\footnote{A charged pion decays to a muon or antimuon,
which subsequently decay to an electron or positron as well as neutrinos.} and the neutral pion to
gamma rays. Due to the low density of the ICM ($n_{\mathrm{ICM}}\sim 10^{-3}$ cm$^{-3}$), the large
scales of a cluster and the volume filling magnetic fields in the ICM, CR hadrons will be confined in
the cluster on timescales comparable to or longer than the Hubble time \citep[][]{article:Volk_etal:1996,
article:Berezinsky_etal:1997} allowing them to accumulate over time. Depending on the CR pressure,
quantified by the CR-to-thermal pressure fraction
$X_\CR=\expval{P_{\CR}}/\expval{P_{\mathrm{th}}}$, an observable flux of high-energy gamma rays
can be expected.

A substantial value of $X_\CR$ can bias hydrostatic estimates of cluster masses as well as the
temperature decrement of the cosmic microwave background (CMB) in the direction of a galaxy cluster
due to the Sunyaev-Zel'dovich effect. This could then severely jeopardize the use of clusters to
determine cosmological parameters. Comparing X-ray and optical potential profiles in the centers of
galaxy clusters yields an upper limit of 20-30 \% of non-thermal pressure (that can be composed of
CRs, magnetic fields or turbulence) relative to the thermal gas pressure
\citep{article:Churazov_etal:2008, article:Churazov_etal:2010}. An analysis that compares spatially
resolved weak gravitational lensing and hydrostatic X-ray masses for a sample of 18 galaxy clusters
detects a deficit of the hydrostatic mass estimate compared to the lensing mass of $20 \%$ at
$R_{500}$ -- the radius within which the mean density is 500 times the critical density --
suggesting again a substantial non-thermal pressure contribution on large scales
\citep{article:Mahdavi_etal:2008}. Observing high-energy gamma-ray emission is a complementary
method of constraining the pressure contribution of CRs that is most sensitive to the cluster core
region. However, it assumes that the CR component is fully mixed with the ICM and may not a
two-phase structure of CRs and the thermal ICM. A fraction $X_\CR$ of only a few percent is
required in order to produce a gamma-ray flux observable with current generation gamma-ray
telescopes rendering this technique at least as sensitive as the dynamical and hydrostatic method
(which are more general in that they are sensitive to any non-thermal pressure component).

Gamma-ray emission can also be produced by Compton up-scattering of ambient photons, for example
CMB photons, on ultra-relativistic leptons. Those leptons can either be secondaries from the above
mentioned CR interactions or be injected into the ICM by powerful cluster members and be further
accelerated by diffusive shock acceleration or turbulent reacceleration processes
\citep[][and references therein]{article:SchlickeiserSieversThiemann:1987}.

A third mechanism for gamma-ray production in a galaxy cluster is self-annihilation of a DM
particle, e.g. a weakly interacting massive particle (WIMP). As already mentioned, about 80\% of
the cluster mass is in the form of dark matter. The gamma-ray emissivity from DM annihilations is
proportional to the square of the DM density integrated along the line of sight, making galaxy clusters
potentially good candidates for DM searches \citep{article:EvansFerrerSarkar:2004,
article:BergstromHooper:2006, article:PinzkePfrommerBergstrom2009, article:Cuesta_etal:2011},
despite their large distances compared to other common targets for DM searches, such as dwarf
spheroidal galaxies \citep{article:Strigari_etal:2007, article:Acciari_etal:2010,article:Aliu_etal:2009}
or the Galactic center \citep{article:Kosack_etal:2004, article:Aharonian_etal:2006,
article:Aharonian_etal:2009b}.

While several observations of clusters of galaxies have been made with satellite-borne and
ground-based gamma-ray telescopes, a detection of gamma-ray emission from a cluster has yet to be
made. Observations with EGRET \citep{article:Sreekumar_etal:1996, article:Reimer_etal:2003} and the
Fermi Gamma-ray Space Telescope \citep{article:Ackermann_etal:2010} have provided upper limits on the
gamma-ray fluxes (typically $\sim10^{-9}$ ph cm$^{2}$ s$^{-1}$ for Fermi observations) for several
galaxy clusters in the MeV to GeV band. Upper limits on the very-high-energy (VHE) gamma-ray flux
from a few select clusters, including the Coma cluster, have been provided by observations with
ground-based imaging atmospheric Cherenkov telescopes \citep[IACTs;][]{article:Perkins_etal:2006,
inproc:Perkins_etal:2008, article:Aharonian_etal:2009a, article:Aleksic_etal:2010}.

The Coma cluster of galaxies (ACO 1656) is one of the most thoroughly studied clusters over all
wavelengths \citep{article:Voges_etal:1999}. Located at a distance of about 100 Mpc
\citep[$z=0.023$;][]{article:StrubleRood:1999}, it is one of the closest massive clusters
\citep[$M \sim10^{15}M_{\odot}$;][]{article:Smith:1983, article:Kubo_etal:2008}. It hosts both a
giant radio halo \citep{article:Giovannini_etal:1993,article:Thierbach_etal:2003} and peripheral
radio relic, which appears connected to the radio halo with a ``diffuse'' bridge 
\citep[see discussion in][]{article:BrownRudnick:2010}. It has been suggested
\citep{article:Ensslin_etal:1998} and successively demonstrated by cosmological simulations, which
model the non-thermal emission processes, \citep{article:PfrommerEnsslinSpringel:2008,
article:Pfrommer:2008, article:Battaglia_etal:2009, article:Skillman_etal:2011}, that the relic is
an infall shock. Extended soft thermal X-ray (SXR) emission is evident from the ROSAT all-sky
survey in the 0.1 to 2.4 keV band \citep{article:BrielHenryBohringer:1992}. Observations with
XMM-Newton \citep{article:Briel_etal:2001} revealed substructure in the X-ray halo supported by
substantial turbulent pressure of at least $\sim 10 \%$ of the total pressure
\citep{article:Schuecker_etal:2004}. The Coma cluster is a natural candidate for gamma-ray
observations.

In this article, results from the VERITAS observations of the Coma cluster of galaxies are
reported, with complementing analysis of available data from the Fermi Gamma-ray Space Telescope.
The VERITAS and Fermi data have been used to place constraints on cosmic-ray particle populations,
magnetic fields, and dark matter in the cluster. Throughout the analyses, a present day Hubble
constant of $H_{0} = 100h$ km s$^{-1}$ Mpc$^{-1}$ with $h=0.7$ has been used.

%
%

\section{VERITAS Observations and Analysis}
The VERITAS gamma-ray detector \citep{article:Weekes_etal:2002} is an array of four 12 m diameter
imaging atmospheric Cherenkov telescopes \citep[IACTs;][]{article:Holder_etal:2006} located at an
altitude of 1268 m a.s.l. at the Fred Lawrence Whipple Observatory in southern Arizona
(31$^{\circ}$~40\arcmin~30\arcsec~N, 110$^{\circ}$~57\arcmin~07\arcsec~W). Each of the telescopes
is equipped with a 499 pixel camera covering a 3.5$^{\circ}$ field of view. The array, completed in
the spring of 2007, is designed to detect gamma-ray emission from astrophysical objects in the
energy range from 100 GeV to more than 30 TeV. The energy resolution is about 15\% and the angular
resolution (68\% containment) is about 0.1$^{\circ}$ per event at 1 TeV. The sensitivity of the
array allows for detection of a point source with a flux of 1\% of the steady Crab Nebula flux
above 300 GeV at the $5\sigma$ level in under 30 hours.\footnote{The integral flux sensitivity
above 300 GeV was improved by about 30\% with the move of one telescope in the summer of 2009. The
move had no impact on this data set.}

The Coma cluster was observed with VERITAS between March and May in 2008 with all four telescopes
fully operational. The total exposure amounts to 18.6 hours of quality-selected live time, i.e.
time periods of astronomical darkness, clear sky conditions and without technical problems with the
array. The center of the cluster was tracked in \emph{wobble} mode, where the expected source
location is offset from the center of the field of view, to allow for simultaneous background
estimation \citep{article:Formin_etal:1994}. All of the observations were made in a small range of
zenith angles averaging $\sim 21^{\circ}$.

The data analysis was performed following the standard VERITAS procedures described in
\citet{inproc:Cogan_etal:2007} and \citet{inproc:Daniel_etal:2007}. Prior to event reconstruction
and selection, all shower images are calibrated and cleaned. Showers are then reconstructed for
events with at least two contributing telescope images that pass the following quality selection
criteria: more than four participating pixels in the camera, number of photoelectrons in the image
is larger than 75, and the distance from the image centroid to the center of the camera is less
than $1.43^{\circ}$. These quality selection criteria impose an energy threshold\footnote{Defined
as the energy corresponding to the maximum of the product function of the observed spectrum with
the collection area} of about 220 GeV. In addition, events for which only images from the two
closest-spaced telescopes\footnote{In the array configuration prior to summer 2009, two telescopes 
had a separation of only 35 m.} survive quality selection are rejected, as they introduce an
irreducible high background rate, due to local muons, degrading the instrument sensitivity
\citep{article:MaierKnapp:2007}.

Gamma-ray like events are separated from the CR background by imposing cuts on scaled parameters
\citep{article:Aharonian_etal:1997, article:Krawczynski_etal:2006} calculated from a parametrized
moment analysis of the shower images \citep{inproc:Hillas:1985}. The gamma-hadron separation cuts
used in this analysis were optimized a priori for a weak point source (3\% Crab Nebula flux level)
and a photon index of 2.4 using data taken on the Crab Nebula during the same epoch. Because the
VHE gamma-ray spectrum for the Coma cluster is expected to be hard, photon index of about 2.3,
these cuts are suitable for the analysis of the Coma cluster data set. 

The Coma cluster is a very rich cluster of galaxies with many plausible sites for gamma-ray
emission: the core region, the peripheral radio relic, as well as individual powerful cluster
member galaxies. VERITAS has a large enough field of view to allow investigation of several of
these scenarios. In this work the focus has been on the core region and three cluster members. The
latter are treated as point sources in the analysis, whereas the core is treated as either a point
source or a mildly extended source with intrinsic source radius $0.2^{\circ}$ or $0.4^{\circ}$,
similar to the extension of the thermal soft X-ray emission from the core. There
is evidence of a recent merger event between the two central galaxies NGC 4889 and NGC 4874
\citep{article:Tribble:1993}. There is also evidence for an excess of non-thermal X-ray emission
from these galaxies as well as the galaxy NGC 4921 \citep{article:Neumann_etal:2003}. Therefore,
searches for point-like VHE gamma-ray emission have been conducted at the locations of these
galaxies. The regions of interest considered in this work are summarized in Table \ref{table:roi}.

The ring background model \citep{article:Aharonian_etal:2001} is used to estimate the background
due to CRs misinterpreted as gamma rays (the cuts described above reject more than 99\% of all
CRs). The total number of events in a given region of interest is then compared to the estimated
background from the off-source region scaled by the ratio of the solid angles to produce a final
excess. The VHE gamma-ray significance is then calculated using the method by
\citet{article:LiMa:1983}. Significance skymaps over the VERITAS field of view produced with a
$0.2^{\circ}$ sampling radius are shown in Figure \ref{fig:skymaps} with overlaid X-ray and radio
contours from the ROSAT all-sky survey \citep{article:BrielHenryBohringer:1992}  and GBT 1.4 GHz
observations \citep{article:BrownRudnick:2010} respectively.

%
%

\subsection{VERITAS Results}
No significant excess of VHE gamma-rays from the Coma cluster was detected with VERITAS, as
illustrated by the $\theta^{2}$ distribution shown in Figure \ref{fig:thetasq}. The $\theta^{2}$
distribution is a radial comparison of the ON-source event distribution with the estimated
background. It extends out to a radius of $0.42^{\circ}$ to cover all regions of interest
considered in this work. A 99\% confidence level upper limit is calculated for each region of
interest using the method described by \citet{article:Rolke_etal:2005} assuming a Gaussian
background. A lower bound of zero is imposed on the signal rate to handle the negative excesses
seen for several of the regions of interest. Figure \ref{fig:sigdist} shows that the VERITAS data
is well fit by a Gaussian with a mean close to zero and a standard deviation within a few percent
of unity. 

Table \ref{table:results} lists the upper limits for the selected regions of interest tabulated in
Table \ref{table:roi}. These upper-limit calculations depend on the gamma-ray spectrum, which in
this work is assumed to be a power law with a spectral index $\alpha=2.1, 2.3$, and $2.5$.

%
%

\section{Fermi-LAT Analysis and Results}
The Large Area Telescope (LAT) on board the \emph{Fermi} Gamma-ray Space Telescope (\emph{Fermi})
has observed the Coma cluster in all-sky survey mode since its launch in June 2008. The
\emph{Fermi}-LAT is sensitive to gamma rays in the 20 MeV to $\sim$300 GeV energy range and is thus
very complimentary to the VERITAS observations. \citet{article:Ackermann_etal:2010} reported on the
search for gamma-ray emission from thirty-three galaxy clusters in the data from the first 18
months, including the Coma cluster, for which an upper limit of $4.58\times 10^{-9}$ ph. cm$^{-2}$
s$^{-1}$ in the 0.2 to 100 GeV energy band was reported. This limit is expected to improve as the
exposure is increased. In this work an updated analysis which includes additional data taken during
the period 2010 February 5 to 2011 June 10 is presented as a compliment to the VERITAS results.

The analysis of this work follows the same procedure as described in detail in
\citet{article:Abdo_etal:2009}. To eliminate albedo gamma rays from the Earth limb, a zenith angle
cut of 105$^{\circ}$ is applied. The \emph{Fermi} Science Tools version v9r23p1 and Instrumental
Response Functions (IRFs) P6\_V11\_Diffuse (a model of the spatial distribution of photon events
calibrated  in flight) were used throughout this work. Only photons above 1 GeV are considered,
where the constraints on the flux are most meaningful. The \emph{Fermi}-LAT collaboration estimates
the systematic uncertainties at 10 GeV to be around 20\%.
\footnote{\url{http://fermi.gsfc.nasa.gov/ssc/data/analysis/LAT\_caveats.html}} Flux upper limits
at the 99\% confidence level in the energy range 1--30 GeV, assuming both a unresolved, point-like
or spatially extended emission, are tabulated in Table \ref{table:fermi}. The gamma-ray emission is
assumed to follow a power-law distribution with a photon index $\alpha=2.1$, 2.3, and $2.5$.

%
%

\section{Gamma Ray Emission from Cosmic Rays}
We decided to adopt a multi-faceted approach to constrain the CR-to-thermal pressure distribution
in the Coma cluster using the upper limits dereived from VERITAS and \emph{Fermi}-LAT data in this
work. This approach includes (1) a simplified multi-frequency analytical approach that assumes a
constant CR-to-thermal energy density and a power-law spectrum in momentum, (2) an analytic model
derived from cosmological hydrodynamical simulations of the formation of galaxy clusters, and (3)
an approach that uses the
observed luminosity of the giant radio halo in Coma to place a lower limit on the expected
$\gamma$-ray flux in the hadronic model -- where the radio-emitting electrons are secondaries from
CR interactions and which is independent of the magnetic field distribution.  This translates into a
minimum CR pressure which, if challenged by tight gamma-ray limits/detections, enables scrutinizing
the hadronic interaction model of the formation of giant radio halos. Alternatively, by realizing a
spatial CR distribution that is just consitent with the flux upper limits and requiring the model to
match the observed radio data, this enables us to derive a lower limit on the magnetic field
distribution. We stress again that his approach assumes the validity of the hadronic interaction
model. Modelling the CR distribution through different techniques enable
us to bracket our ignorance about the underlying plasma physics that shape the CR distribution and
hence reflect the Bayesian priors that are imposed on the modelling \citep[see][for a
discussion]{article:PinzkePfrommerBergstrom}.

%
%

\subsection{Simplified analytical model}
\label{sec:simple}

We start by adopting a simplified analytical model that assumes a power-law CR spectrum and a
constant relative CR pressure, i.e., we adopt the {\em isobaric model of CRs} following the approach
of \citet{article:PfrommerEnsslin:2004b}. To be independent of additional assumptions and in line
with earlier work in the literature, we do not impose a low-momentum cutoff on the CR distribution
function, i.e., we adopt $q=0$. Since a priori, the CR spectral index is unconstrained, we vary it in
the range $2.1<\alpha_{\rmn{p}}<2.5$, which is compatible with the
radio spectral index of the giant radio halo of the Coma cluster after accounting for the spectral
steepening at frequencies $\nu\sim5~\rmn{GHz}$ due to the Sunyaev-Zel'dovich effect
\citep{article:PfrommerEnsslin:2004b}.\footnote{Assuming a magnetic field of 1 $\mu$G, the CR protons
responsible for the GHz radio emitting electrons have an energy of $\sim100$ GeV and are $\sim$ 20
times less energetic than those CR protons responsible for 200-GeV gamma-ray emission.} To model
the thermal pressure, we adopt the electron density profile for the Coma cluster that has been
inferred from ROSAT X-ray observations \citep{article:BrielHenryBohringer:1992} and use a constant
temperature of $kT= 8.25$~keV throughout the virial region.

Table \ref{table:constraints_simple} shows the resulting constraints on the relative CR pressure, $X_{\CR}
= \expval{P_{\CR}}/\expval{P_\rmn{th}}$, averaged within the virial radius, $R_\rmn{vir}=2.2$~Mpc,
that we define as the radius of a sphere enclosing a mean density that is 200 times the critical
density of the Universe. The VERITAS flux upper limits strongly depend on $\alpha$ due to the
comparably large lever arm from GeV energies (that dominate the CR pressure) to TeV-energies. The
flux measurements within 0.2$^{\circ}$ are the most constraining and yield limits on $X_\CR$
between 0.048 and 0.43
(for $\alpha$ varying between 2.1 and 2.5), with a constraint of $X_\CR<0.1$ for $\alpha=2.3$ (close
to the spectral index predicted by the simulations of \citet{article:PinzkePfrommer:2010} around 220
GeV).  The {\em Fermi} limits depend only weakly on $\alpha$ and are most constraining for an
aperture of 0.4$^{\circ}$.  The best limit of $X_\CR< 0.017$ is achieved for $\alpha=2.3$ and only
slightly worse for $\alpha\simeq 2.1$ yielding $X_\CR<0.025$.

%
%

\subsection{Simulation-based approach}
We complement the simplified analytical analysis with a more realistic and predictive approach
derived from cosmological hydrodynamical simulations. We adopt the universal spectral and spatial
gamma-ray model developed by \citet{article:PinzkePfrommer:2010} to estimate the emission from
decaying neutral pions which in clusters dominates over the inverse Compton (IC) emission above
100 MeV. Given a density profile as, e.g., inferred by
cosmological simulations or X-ray observations, the analytic approach models the CR distribution and
the associated radiative emission processes from radio to the gamma-ray band. This formalism was
derived from high-resolution simulations of clusters of galaxies that included radiative hydrodynamics,
star formation, supernova feedback, and followed CR physics by tracing the most important injection
and loss processes self-consistently while accounting for the CR pressure in the equation of motion
\citep{article:PfrommerSpringelEnsslinJubelgas, article:EnsslinPfrommerSpringelJubelgas:2007,
  article:JubelgasSpringelEnsslinPfrommer:2008}. The results are in line with earlier numerical
results on some of the overall characteristics of the CR distribution and the associated radiative
emission processes \citep{article:DolagEnsslin:2000, article:MiniatiRyuKangJones:2001,
  article:Miniati:2003, article:Pfrommer_etal:2007, article:PfrommerEnsslinSpringel:2008,
  article:Pfrommer:2008}.
 
The overall normalization of the CR and gamma-ray distribution scales non-linearly with the
acceleration efficiency at structure formation shock. Following recent observations at
supernova remnants \citep{article:Helder_etal:2009} as well as theoretical studies
\citep{article:KangJones:2005}, we adopt an optimistic but nevertheless realistic value of this
parameter and assume that 50\% of the dissipated energy at strong shocks is injected into CRs while
this efficiency rapidly decreases for weaker shocks. Since the vast majority of internal formation
shocks (merger and flow shocks) are weak shocks with Mach numbers $M\lesssim3$
\citep[e.g.,][]{article:Ryu_etal:2003}, they do not significantly contribute to the CR population in
clusters. Instead, strong shocks during the formation epoch of clusters and strong accretion shocks
at the present time (at the boundary of voids and filaments/super-cluster regions) dominate the
injection of CRs which subsequently are adiabatically transported through the cluster. Hence, the model
provides a plausible upper limit for the CR contribution from structure formation shocks
in galaxy clusters which can be scaled with the effective acceleration efficiency. Other possible CR
sources, such as AGN and starburst-driven galactic winds have been neglected for simplicity which
could in principle increase the expected gamma-ray yield.

These cosmological simulations only considered advective transport of CRs by turbulent gas motions
which lead to centrally enhanced density profiles.  However, other means of CR transport such as diffusion
and streaming may flatten the CR radial profiles. It has been suggested that advection is dominant
only during mergers, in which case relaxed clusters could have a reduced gamma-ray luminosity by up
to a factor of five compared with merging clusters
\citep{article:EnsslinPfrommerMiniatiSubramanian:2011}. The goal of this and other future work of
seeking for gamma-ray emission from clusters is to constrain the acceleration physics and transport
properties of CRs.

We adopt the density profile of thermal electrons as discussed in \S \ref{sec:simple} and model the
temperature profile
of the Coma cluster with a constant central temperature of $kT= 8.25$~keV and a characteristic
decline toward the cluster periphery in accordance with a fit to the universal profile derived from
cosmological cluster simulations \citep{article:PinzkePfrommer:2010, article:Pfrommer_etal:2007}
and the behavior of a nearby sample of deep {\em Chandra} cluster data
\citep{article:Vikhlinin_etal:2005}. This enables us to adopt the spatial and spectral distribution
of CRs according to the model by \citet{article:PinzkePfrommer:2010} that neglects the contribution
of cluster galaxies.

Figure \ref{fig:spectrum} shows the expected integral spectral energy distribution of Coma within
the virial radius (dotted lines). Additionally shown are integrals of the differential spectrum for
finited energy intervals across the angular apertures tested in this study (solid lines). These
model fluxes are compared to {\em Fermi} and VERITAS flux upper limits of the same energy intervals,
summarized in Table \ref{table:constraints}.  The gamma-ray flux limit of {\em Fermi} in the energy
interval of 1--3 GeV ($<0.4^\circ$) is most constraining but still above the model predictions that
assume an optimistically large shock acceleration efficiency and CR transport parameters as laid
out above.  In our further analysis, we use the most constraining {\em Fermi} flux limits in the
energy interval of 1-3 GeV as well as the gamma-ray flux limits of VERITAS in the energy range above
220 GeV.

Figure \ref{fig:xcr} shows the relative CR pressure, $X_{\CR} =
\expval{P_{\CR}}/\expval{P_\rmn{th}}$ as a function of radial distance, $R$, from the Coma cluster
center (red lines) and contained within $R$ (blue lines). All radii are shown in units of the virial
radius, $R_\rmn{vir}=2.2$~Mpc. To compute the CR pressure, we assume a low-momentum cutoff of the CR
distribution at $q = 0.8\,m_p c$ to reflect tge high cooling rates at low CR energies. The relative CR
pressure rises toward the outer regions on account of the higher efficiency of CR acceleration at the
peripheral accretion shocks compared to the weak central flow shocks. Adiabatic compression of a
mixture of CRs and thermal gas disfavours the
CR pressure relative to the thermal pressure on account of the softer equation of state of CRs. The
weak increase of $X_{\CR}$ toward the core is due to the comparably fast thermal cooling of gas.

In the case of VERITAS, for the most constraining regions tested (within an aperture of radius
0.2$^{\circ}$), the predicted CR pressure is a factor of 7.2 below the inferred upper limits of VERITAS
(see Table
\ref{table:results} and assuming a spectral index of $\alpha=2.3$ which matches the simulated one at
energies $E_\gamma=200$ GeV). To first order, we can scale the averaged relative CR pressure of
our model by that
factor, keep the spatial behavior and obtain an integrated limit of the relative CR pressure of
$X_{\CR}<0.105$ within 0.2$^{\circ}$ that translates to a limit within the cluster virial radius of
$X_\CR<0.152$ (solid lines of Figure \ref{fig:xcr}).  This limit is less constraining by 50\% in
comparison to the simplified analytical model, which gives $X_\CR<0.1$. This difference is
explained by the concave curvature of the simulated spectrum which accumulates additional pressure
toward GeV energies in comparison to a pure power-law spectrum of $\alpha=2.3$.

Equivalently, the most constraining {\em Fermi} upper limit in the energy interval of 1-3 GeV
($<0.4^\circ$) is a factor of 1.1 larger than our model predictions (assuming $\alpha=2.1$ which is
very close to the simulated spectral index for the energy range of 1-3 GeV). Scaling our integrated
relative CR pressure profile yields the constraint of $X_{\CR}<0.018$ within 0.4$^{\circ}$ that
translates to a limit within the cluster virial radius of $X_\CR<0.024$ (dashed lines of Figure
\ref{fig:xcr}).  The $X_\CR$-constraint evaluated within the cluster virial radius is comparable to
the constraint in our simplified model of $X_\CR<0.025$.  Naturally, with the Fermi limits we probe
the region around GeV-energies that dominate the CR pressure and we do not expect any differences in
comparison to the simplified power-law model in comparison to the universal CR spectrum with its
concave CR spectrum found in the simulations.

%
%

\subsection{Minimum gamma-ray flux}
\label{sec:Fmin}
For clusters that host radio halos we can derive a minimum gamma-ray flux in the hadronic
model of radio halos -- where the radio-emitting electrons are secondaries from CR interactions. 
Hadronic interactions channel about the same power into secondary electrons and $\pi^{0}$-decay
gamma rays. A stationary distribution of CR electrons loses all its energy to
synchrotron radiation for strong magnetic fields ($B \gg B_{\rmn{CMB}} \simeq 3.2 \mu$G), and thus the
ratio of gamma-ray to synchrotron flux becomes independent of the spatial distribution of CRs and
thermal gas \citep{article:Voelk:1989, article:Pohl:1994, article:Pfrommer:2008}, in particular with
$\alpha_{\nu}\simeq 1$ as the observed synchrotron spectral index.  Hence we can derive a minimum
gamma-ray flux in the hadronic model
\begin{equation}
\label{eq:Fmin}
\mathcal{F}_{\gamma,\rmn{min}} = \frac{\dps A_{\gamma}}{\dps A_{\nu}}\frac{\dps L_{\nu}}{\dps 4\pi D_{\rmn{lum}}^{2}},
\end{equation}
where $L_{\nu}$ is the observed luminosity of the radio mini-halo, $D_{\rmn{lum}}$ denotes the
luminosity distance to the respective cluster, $A_\gamma$ and $A_\nu$ are dimensional constants
that depend on the hadronic physics of the interaction \citep{article:Pfrommer:2008,
Pfrommer_etal:2008}. Lowering the magnetic field would require an increase in the energy density of
CR electrons to reproduce the observed synchrotron luminosity and thus increase the associated
gamma-ray flux.

To derive a minimum gamma-ray flux that is comparable to the upper limits, we need to determine the
radio flux within the corresponding angular regions. To this end, we fit the point-source
subtracted, azimuthally averaged radio halo profile at 1.38 GHz \citep{article:Deiss_etal:1997}
with a $\beta$-model,
\begin{equation}
\label{beta}
 S_{\nu} (r_{\bot})= S_{0} \left[ 1 + \left( \frac{r_{\bot}}{r_{\rmn{c}}}\right)^{2}\right]^{-3\beta + 1/2},
\end{equation}
where $S_{0} = 1.1 \times 10^{-3}\,\rmn{Jy\,arcmin}^{-2}$, $r_{\rmn{c}} = 450$ kpc, and
$\beta = 0.78$.  Within the error bars, this profile is consistent with 326-MHz data taken by
\citet{article:Govoni_etal:2001} when scaled with a radio spectral index of 1.15. 

The results for the minimum gamma-ray flux $F_{\rmn{min}}(>220~\rmn{GeV})$ and the minimum relative
CR pressure, $X_{\CR,\,\rmn{min}} = F_{\rmn{min}}/\tilde{F}$ are shown in Table
\ref{table:constraints}, where $F_\rmn{iso} = X_\CR\,\tilde{F}$ is the gamma-ray flux in the simplified
model introduced in \S\ref{sec:simple}. Even in the most constraining cases,
these are a factor of about 60 below the VERITAS upper limits (adopting $\alpha=2.1$,
$<0.2^{\circ}$) and a factor of about 20 below the {\em Fermi} upper limits (adopting $\alpha=2.3$,
$<0.4^{\circ}$).

%
%
 
\subsection{Constraining the Magnetic Field}
\label{sec:B}
In the previous section we have obtained an absolute lower limit in the gamma-ray emission in
the hadronic model by assuming high magnetic fields, $B\gg B_\rmn{CMB}$. We can turn the
argument around and use our upper limit on the gamma-ray emission (and by extension on the
CR pressure) to infer a lower limit on the magnetic field needed to explain the observed radio
emission. This, again, assumes the validity of the hadronic model of radio halos, in which the
radio-emitting electrons are secondaries from CR interactions. Lowering the gamma-ray
constraint will tighten the magnetic field limit. In case of a conflict with magnetic field
measurements by other methods, e.g., Faraday rotation measure (RM),\footnote{Generally,
Faraday RM analysis of the magnetic field strength by, e.g., background sources through
clusters are degenerate with the magnetic coherence scale and may be biased by the unknown
correlation between magnetic and density fluctuations.} the hadronic model of radio halos would
be challenged. The method we use to constrain the magnetic field inherits a dependence on
the assumed radial scaling which we parameterize as
\begin{equation}
\label{eq:B}
B(r) = B_{0} \,\left(\frac{n_{\rmn{e}}(r)}{n_{\rmn{e}}(0)}\right)^{\alpha_B},
\end{equation}
as suggested by Faraday RM studies and numerical magneto-hydrodynamical (MHD) simulations
\citep[][and references therein]{article:Bonafede_etal:2010, article:Bonafede_etal:2011}. In fact,
the magnetic field in the Coma cluster is among the best constrained, because its proximity permits
RM observations of seven radio-sources located at projected distances of 50 to 1500 kpc from the
cluster center. The best-fit model yields $B_{0} = 4.7^{+0.7}_{-0.8}\,\mu$G and
$\alpha_{B} = 0.5^{+0.2}_{-0.1}$ \citep{article:Bonafede_etal:2010}. We aim for constraining the
central field strength, $B_{0}$, and permit the magnetic decline, $\alpha_{B}$, to vary
within a reasonable range of $\Delta\alpha_{B}=0.2$ as suggested by the Faraday RM studies.
We proceed as follows:

\begin{enumerate}
\item
Given a model for the magnetic field with $\alpha_B$ and an initial guess for $B_0$, we determine
the profile of the relative CR pressure, $X_{\CR}(r)$, by matching the hadronically produced
synchrotron emission to the observed radio halo emission over the entire extent. To 
this end, we de-project the fit to the surface brightness profile of Eqn.~\ref{beta} (using an Abel
integral equation, see Appendix of \citealt{article:PfrommerEnsslin:2004b}) yielding the radio
emissivity,

\begin{equation}
\label{eq:Coma:radio}
j_{\nu} (r) = \frac{S_{0}}{2\pi\, r_{\rmn{c}}}\,
\frac{6\beta - 1}{\left(1 + r^{2}/r_{\rmn{c}}^{2}\right)^{3 \beta}}\,
\mathcal{B}\left(\frac{1}{2}, 3\beta\right)
= j_{\nu,0} \left(1 + r^2/r_{\rmn{c}}^{2}\right)^{-3 \beta},
\end{equation}
where $\mathcal{B}$ denotes the beta function. It is generically true for weak magnetic fields
($B<B_{\rmn{CMB}}$) in the outer parts of the Coma halo that the product $X_{\CR}(r)X_{B}(r)$
(where $X_B$ denotes the magnetic-to-thermal energy density) has to increase by a factor of about
100 toward the radio halo periphery to account for the observed extent. If we were to adopt a
steeper magnetic decline such as $\alpha_{B}=0.5$ which produces a flat $X_{B}(r)$, the relative CR
pressure would have to rise accordingly by a factor of 100.

\item
Given this realization for $X_{\CR}$, we compute the pion-decay gamma-ray surface brightness
profile, integrate the flux within a radius of $(0.13, 0.2, 0.4)$ degree, and scale the CR profile
in order to match the corresponding VERITAS/{\em Fermi} flux upper limits. This scaling factor,
$X_{\CR,0}$, depends on the CR spectral index, $\alpha$, (assuming a power-law CR population for
simplicity), the radial decline of the magnetic field, $\alpha_{B}$, and our initial guess for
$B_{0}$.

\item
We then solve for $B_{0}$ while matching the observed synchrotron profile while fixing the profile
of $X_{\CR}(r)$ as determined through the previous two steps. Note
that for $B_{0} \gg B_{\rmn{CMB}}$ and a radio spectral index of $\alpha_{\nu}=1$, the solution
would be degenerate.

\item
Inverse Compton cooling of CR electrons on CMB photons introduces a characteristic scale of
$B_{\rmn{CMB}}\simeq 3.2\,\mu$G which imprints as a non-linearity on the synchrotron emissivity as a
function of magnetic field strength. Hence we have to iterate through the previous steps until our
solution for the minimum magnetic field $B_{0}$ is converged.
\end{enumerate}

Table \ref{table:Bmin} shows the resulting lower limit of the central magnetic field that
range from $B_{0} = 0.5$ to $1.4\,\mu$G in case the of VERITAS and from
$B_{0} = 1.1$ to $4.6\,\mu$G in the case of \emph{Fermi}.\footnote{Note that a central magnetic
field of $3\,\mu$G corresponds 
in the Coma cluster to a relative magnetic energy density of $X_B=0.005$.} Since these lower limits
on $B_{0}$ are below the values favored by Faraday RM for the entire parameter space spanned by
$\alpha_{B}$ and $\alpha$, the hadronic model is still viable. Future gamma-ray observations of
the Coma cluster may put more stringent constraints on the parameters of the hadronic model.

A few remarks are in order. (1) For the VERITAS limits, the hardest
CR spectral indices correspond to the tightest limits on $B_{0}$, because the CR flux is constrained
around 1 TeV and a comparably small fraction of CRs at 100 GeV would be available to produce
radio-emitting electrons. A high magnetic field would be required to match the observed
synchrotron emission. The opposite is true for the \emph{Fermi} upper limits at 1 GeV, which
probe CRs around a pivot point of 8 GeV: a soft CR spectral index implies a comparably small
fraction of CRs at 100 GeV and hence a strong magnetic field is needed to match the observed
synchrotron flux. (2) For a steeper
magnetic decline (larger $\alpha_{B}$), the CR number density needs to be larger to match the
observed radio emission profiles which would yield a higher gamma-ray flux so that the upper limits
are more constraining. This implies tighter lower limits for $B_{0}$.  (3) Interestingly, in all
cases, the 0.4$^{\circ}$-aperture limits that are the most constraining. For a given magnetic
realization a substantially increasing relative CR pressure profile is needed to match the
observed radio profiles, and therefore that CR realization produces a larger flux within 0.4$^{\circ}$
in comparison with the simplified CR model ($X_{\CR} = \rmn{const.}$). for which the
0.2$^{\circ}$-aperture limits are more constraining in the case of VERITAS. Physically, the large
CR pressure in the cluster periphery may arise from CR streaming into the large available phase
space in the outer regions.

As a final word, in Table~\ref{table:Bmin} we additionally show the corresponding values for the
relative CR pressure (at the largest emission radius at 1~Mpc) such that the model reproduces the
observed radio surface brightness profile. They should be interpreted as upper limits since they
are derived from flux upper limits. For the {\em Fermi} upper limits, they range from 0.1 to 0.4;
hence the $X_{\CR}$-profiles always obey the energy condition, i.e., $P_{\CR} < P_{\mathrm{th}}$
over the entire range of the radio halo emission ($< 1$ Mpc).\footnote{See Figure 3 in
\citet{article:PfrommerEnsslin:2004a} for the entire parameter range assuming minimum energy
conditions and \citet{article:PfrommerEnsslin:2004b}, Figure 7 for a parametrization as adopted in
this study.  We caution, however, that the minimum-energy condition is violated at the outer radio
halo boundary for the range of minimum magnetic-field values inferred by this study.} The
corresponding values for $X_\CR$ in the cluster center are smaller than 0.01 for the entire
parameter space probed in this study. We conclude that the hadronic model is not challenged by
current Faraday RM data and a perfectly viable possibility in explaining the Coma radio halo
emission.

%
%

\section{Emission from Dark Matter Annihilations}
As already mentioned in the introduction, most of the mass in a galaxy cluster is in the form of
DM. While the nature of this DM remains unknown, the most compelling theoretical candidate is a
WIMP. The self-annihilations of WIMPs are expected to produce a flux of secondary gamma rays with a
spectral signature that deviates significantly from the power-law spectrum observed in most
conventional astrophysical sources. The spectrum will also feature a very sharp cutoff at the WIMP
mass. Observations of these spectral characteristics together with the expected difference in spatial
distribution of the gamma-ray intensity compared to conventional astrophysical sources allows for a
an indirect detection DM.

The expected gamma-ray flux due to self-annihilations of WIMPs is given by 
\begin{equation}
\frac{d\Phi_{\gamma}(\Delta\Omega,E)}{dE}=
\frac{\expval{\sigma v}}{8\pi m_{\chi}^{2}}\frac{dN(E,m_{\chi})}{dE} J(\Delta\Omega),
\label{eqn:WIMPflux}
\end{equation}
where $\expval{\sigma v}$ is the thermally averaged product of the total self-annihilation cross
section and the WIMP velocity, $m_{\chi}$ is the WIMP mass, $dN(E,m_{\chi})/dE$ is the differential
gamma-ray yield per annihilation,\footnote{In this work, we have calculated the differential
gamma-ray yield per annihilation using the Pythia Monte Carlo simulator.} $\Delta\Omega$ is the
observed solid angle, and $J$ is the astrophysical factor - a dimensionless factor which determines
the DM annihilation rate and depends on the DM distribution.

Given the upper limit on the observed gamma-ray rate,
$R_{\gamma}(95\%\ \mathrm{CL}) = N_{\gamma}(95\%\ \mathrm{CL}) / T_{\mathrm{obs}}$, we can place
constraints on the WIMP parameter space $(m_{\chi}, \expval{\sigma v})$. From Eq.
(\ref{eqn:WIMPflux}) we get
\begin{equation}
\expval{\sigma v}(95\%\ \mathrm{CL}) > 
R_{\gamma}(95\%\ \mathrm{CL}) \frac{8\pi m_{\chi}^{2}}{J(\Delta\Omega)}
\frac{1}{\int^{m_{\chi}} dE A_{\mathrm{eff}}dN_{\gamma}/dE}
\end{equation}
Because the self-annihilation of a WIMP is a two-body process, the astrophysical factor
$J(\Delta\Omega)$ can be expressed as the line-of-sight integral of the DM density squared
\begin{equation}
J(\Delta\Omega)=\left(\frac{1}{\rho_{c}^{2}R_{H}}\right)
\int_{\Delta\Omega}d\Omega\int d\lambda \rho^{2}(\lambda,\Omega),
\end{equation}
where $\lambda$ represents the line-of-sight. Here,
$J(\Delta\Omega)$ has been normalized to the square of the critical density, $\rho_{c}$, and the
Hubble radius, $R_{H}$. In this work we have adopted the DM density profile by
Navarro-Frenk-White \citep[NFW;][]{article:NavarroFrenkWhite:1997},
\begin{equation}
\rho_{\chi}(r)=\rho_{s}\left(\frac{r}{r_{s}}\right)^{-1}\left(1+\frac{r}{r_{s}}\right)^{-2},
\end{equation} 
which parametrizes the distribution of DM by the scale radius $r_{s}$ and the scale density
$\rho_{s}$. Using weak-lensing measurements of the virial mass in the Coma cluster
\citep{article:Gavazzi_etal:2009} we find for these parameters $r_{s}=0.654$ Mpc and
$\rho_{s}=4.4\times 10^{4}$ M$_{\odot}$/Mpc$^{3}$. Table \ref{table:astrofactor} lists the
astrophysical factor calculated for the different VERITAS appertures considered in this work.
It also lists the astrophysical factor calculated for the background region, which has been used
to estimate the gamma-ray contamination due to DM in the background region.

The resulting exclusion regions in the $(\expval{\sigma v}, m_{\chi})$ parameter space are shown in
Figure \ref{fig:dm} for three different DM self-annihilation channels: W$^{+}$W$^{-}$,
b$\bar{\mathrm{b}}$, and $\tau^{+}\tau^{-}$. Depending on the chosen apperture the limits are on
the order of $10^{-20}$ to $10^{-21}$ cm$^{3}$ s$^{-1}$. The numerical values for each of the channels
are listed in Table \ref{table:DMlimits}. We note that these are based on conservative estimates of the
astrophysical factor $J$, as we do not assume boosting by clumping or the Sommerfeld effect.

We also note that when the size of the integration region is increased, the limits on $\expval{\sigma v}$
result from a competition between the gain in the astrophysical factor $\expval{J}$ and the integrated
background. For integration regions larger than 0.2$^{\circ}$ in radius, the astrophysical factor
no longer compensate for the increased count of background events and the signal-to-noise
deteriorates. 



%
%

\section{Discussion and Conclusions}
VERITAS observed the Coma cluster of galaxies for a total of 18.6 hours of high-quality live time
between March and May in 2008. No significant excess of gamma-rays above a threshold of about 220
GeV was detected. Flux upper limits at 99\% confidence level were calculated for the cluster core
and three member galaxies. The core was considered both as a point-like and spatially extended
source. The obtained flux upper limits were used to constrain properties of the cluster.

We employed various approaches to constrain the CR population and magnetic field distribution that
are complementary in their assumptions and hence well suited in assessing underlying Bayesian
priors in the models. (1) We use a simplified ``isobaric CR model'' that is characterized by a
constant CR-to-thermal pressure fraction and has a power-law momentum spectrum.  While this model
is physically not justified a priory, it is simple and widely used in the literature and captures
some aspects of more elaborate models such as (2) the simulation-based analytical approach of
\citet{article:PinzkePfrommer:2010}. This is a ``first principle approach'' that predicts the CR
distribution spectrally and spatially for a given set of assumptions (namely the physics that was
simulated). It is powerful since it only requires the density profile as input due to the
approximate universality of the CR distribution (when neglecting CR diffusion and streaming which
appears, however, necessary to explain the radio halo bimodality). (3) We finally use a pragmatic
approach which models the CR and magnetic distribution such as to reproduce the observed emission
profile of the Coma radio halo. While this approach is also physically not justified, it is
powerful as it shows what the ``correct'' model has to achieve and can point into the direction of
the relevant physics.

Within this pragmatic approach we employ two different directions. First, adopting a high magnetic
field everywhere in the cluster ($B\gg B_\rmn{CMB}$) yields the minimum gamma-ray flux in the
hadronic model of radio halos which we find to be a factor of 20 (60) below the most constraining
flux upper limits of {\em Fermi} (VERITAS). Second, by matching the radio emission profile (i.e.,
fixing the radial CR profile for a given magnetic field model) and by requiring the pion-decay
gamma-ray flux to match the flux upper limits (i.e., fixing the normalization of the CR
distribution) we obtain lower limits on the magnetic field distribution under consideration. Our
limits of the central magnetic field range from $B_{0} = 0.5$ to $1.4\,\mu$G (for VERITAS  flux
limits) and from $B_{0} = 1.1$ to $4.6\,\mu$G (for {\em Fermi} flux limits). Since all these lower
limits on $B_0$ are below the values favored by Faraday RM, $B_{0} = 4.7^{+0.7}_{-0.8}\,\mu$G
\citep{article:Bonafede_etal:2010}, the hadronic model is a very attractive explanation of the Coma
radio halo.

Applying our simplified ``isobaric CR model'' to the most constraining VERITAS limits, we can
constrain the CR-to-thermal pressure, $X_\CR$, to be below 0.048 and 0.43 (for a CR or gamma-ray
spectral index, $\alpha$, varying between 2.1 and 2.5). We obtain a constraint of $X_\CR<0.1$ for
$\alpha=2.3$, the spectral index predicted by simulations at energies around 220 GeV.  This limit
is more constraining by a factor of 1.5 in comparison to the simulation-based model which gives
$X_\CR<0.15$.  This difference is due to the concave curvature of the simulated spectrum which
accumulates additional pressure toward GeV energies in comparison to a pure power-law spectrum of
$\alpha=2.3$.

The {\em Fermi} flux limits constrain $X_\CR$ to be below 0.017--0.025 (for $\alpha$ varying
between 2.3 and 2.1), only weakly depending on the assumed CR spectral index. Assuming $\alpha=2.1$
which is very close to the simulated spectral index for the energy range of 1--3 GeV, we obtain an
almost coincident constraint with our simulation-based model within the virial radius of
$X_\CR<0.024$. That constraint improves to $X_{\CR}<0.018$ for an aperture of 0.4$^\circ$
corresponding to a physical scale of $R \simeq R_{200}/3 \simeq 660$~kpc. These are encouraging
results in that we constrain the CR pressure (of a phase that is fully mixed with the ICM) to be at
most a small fraction of the overall pressure, $<0.025$. As a result, hydrostatic cluster masses
and the total Comptonization parameter due to the Sunyaev-Zel'dovich effect attain at most a very
small bias due to CRs.

\acknowledgments
This research was supported by grants from the U.S. Department of Energy, the U.S. National Science
Foundation, and the Smithsonian Institution, by NSERC in Canada, by Science Foundation Ireland, and
by STFC in the UK. We acknowledge the excellent work of the technical support staff at the FLWO and
the collaborating institutions in the construction and operation of the instrument. C.P. gratefully
acknowledges financial support of the Klaus Tschira Foundation.

%%Facilities: \facility{VERITAS}.

\bibliographystyle{apj}
\bibliography{refs}

\begin{figure*}
\begin{center}
\scalebox{0.45}{\plotone{f1a.eps}}
\scalebox{0.45}{\plotone{f1b.eps}}
\end{center}
\caption{\emph{Left}: Smoothed significance map of the Coma cluster calculated from the observed
excess VHE gamma-ray events over a $4.5^{\circ}\times 4.5^{\circ}$ field of view. The excess counts
were derived using a ring background model \citep{article:Aharonian_etal:2001}. White contours show
the X-ray counts per second in the 0.1 to 2.4 keV energy band from the ROSAT all-sky survey
\citep{article:BrielHenryBohringer:1992}. \emph{Right}: Same as above but with overlaid contours
from GBT radio observations at 1.4 GHz \citep{article:BrownRudnick:2010} where strong point sources
have been subtracted. Shown are also the $0.2^{\circ}$ and $0.4^{\circ}$ radii (dashed cyan)
considered for the extended source analyses presented here.}
\label{fig:skymaps}
\end{figure*}

\begin{figure*}
\begin{center}
\scalebox{1.0}{\plotone{f2.eps}}
\end{center}
\caption{$\theta^{2}$ distribution from VERITAS observations of the Coma cluster of galaxies. The
points with error bars represent the ON-source data sample and the filled area is the estimated
background due to cosmic rays. Each bin represents an annulus around the Coma cluster core
position. The data were derived from the ring background model using a $0.2^{\circ}$ sampling
radius.}
\label{fig:thetasq}
\end{figure*}

\begin{figure*}
\begin{center}
\scalebox{0.75}{\plotone{f3.eps}}
\end{center}
\caption{Distribution of significance for Figure \ref{fig:skymaps} and a sampling radius of
$0.2^{\circ}$. The line is a Gaussian fit to the data, with mean close to zero and almost unit
standard deviation.}
\label{fig:sigdist}
\end{figure*}

\begin{figure*}
\begin{center}
\scalebox{1.0}{\plotone{f4.eps}}
\end{center}
\caption{{\em Figure TBC} Integral flux upper limits for Coma (this work, Tables~\ref{table:results}
  and \ref{table:fermi}) for different integration radii (shown with different colors), assuming
  $\alpha=2.1$. These are compared to integrated spectra of the same energy interval and aperture,
  assuming the universal gamma-ray spectrum of clusters \citep{article:PinzkePfrommer:2010}. To
  guide the eye, we show the underlying universal integral energy distribution of pion decay
  gamma-rays, $E_\gamma F_\gamma(>E_\gamma)$, resulting from hadronic interactions of CRs and ICM
  protons (dotted). Note that the {\em Fermi} limit for the energy interval of 1--3 GeV within the
  aperture of 0.4$^\circ$ is most constraining.}
\label{fig:spectrum}
\end{figure*}

\begin{figure*}
\begin{center}
\scalebox{1.0}{\plotone{f5.eps}}
\end{center}
\caption{{\em Figure TBC} CR-to-thermal pressure, $X_{\mathrm{CR}}=\expval{P_{\CR}}/ \expval{P_{\mathrm{th}}}$, as a
function of radial distance from the center of the Coma. Model predictions are shown by dashed
lines \citep{article:PinzkePfrommer:2010} while these are scaled to match the most constraining
VERITAS upper limits within 0.2$^\circ$ (solid) and {\em Fermi} upper limits within 0.4$^\circ$
(dashed). We compare differential $X_\CR$-profiles (red) to integrated profiles
$X_{\CR}(<R/R_{\rmn{vir}})=\int_0^R P_\CR\, dV / \int_0^R P_{\rmn{th}}\, dV$ (blue) which we use to
compare to the upper limits.}
\label{fig:xcr}
\end{figure*}

\begin{figure*}
\begin{center}
\scalebox{1.0}{\plotone{f6.eps}}
\end{center}
\caption{Limits on the DM annihilation cross section $\expval{\sigma v}$ as a function of the DM
particle mass $m_{\chi}$ derived from the VERITAS gamma-ray flux upper limits presented in this
work.}
\label{fig:dm}
\end{figure*}

\begin{deluxetable}{lcc}
\tablecolumns{3}
\tablewidth{0pc}
\tablecaption{Regions of interest in the Coma cluster field of view}
\tablehead{
\colhead{Source} &
\colhead{RA} &
\colhead{Dec}
}
\startdata
Core & \RA{12}{59}{48.7} & \Dec{+27}{58}{50.0}\\
NGC 4889 & \RA{13}{00}{08.13} & \Dec{+27}{58}{37.03}\\
NGC 4874 & \RA{12}{59}{35.71} & \Dec{+27}{57}{33.37}\\
NGC 4921 & \RA{13}{01}{26.12} & \Dec{+27}{53}{09.59}\\
\enddata
\tablecomments{The cluster core is considered both as a point source and a modestly extended
source. Three central galaxies are also considered in point-source searches. The choice is based on
evidence of an excess of non-thermal X-ray emission \citep{article:Neumann_etal:2003} at the
location of these galaxies.}
\label{table:roi}
\end{deluxetable}

\begin{deluxetable}{lccccccccc}
\tablecolumns{10}
\tablewidth{0pc}
\tablecaption{VERITAS flux upper limits for different regions of interest in the Coma cluster of
galaxies and surroundings}
\rotate
\tablehead{
\colhead{Source} & 
\colhead{RoI\tablenotemark{a} [deg]} & 
\colhead{$N_{S}$} & 
\colhead{S\tablenotemark{b} [$\sigma$]} &
\multicolumn{6}{c}{Flux U.L.\tablenotemark{c}} \\
\colhead{} & 
\colhead{} & 
\colhead{} & 
\colhead{} &
\multicolumn{2}{c}{$\alpha=2.1$} &
\multicolumn{2}{c}{$\alpha=2.3$} &
\multicolumn{2}{c}{$\alpha=2.5$}
}
\startdata
Core & 0 & 17 & 0.84 & 2.59 & (0.78\%) & 2.78 & (0.83\%) & 2.97 & (0.89\%) \\
& 0.2 & -41 & -1.0 & 1.96 & (0.59\%)  & 2.09 & (0.63\%) & 2.21 & (0.66\%) \\
& 0.4 & -26 & -0.30 & 4.44 & (1.3\%)  & 4.74 & (1.4\%) & 5.02 & (1.5\%)\\
NGC 4889 & 0 & 3 & 0.14 & - & - & 1.85 & (0.55\%)  & - & - \\
NGC 4874 & 0 & -14 & -0.71 & - & - &  1.51 & (0.45\%)  & - & - \\
NGC 4921 & 0 & -4 & -0.23 & - & - &  2.41 & (0.72\%)  & - & -
\enddata
\tablenotetext{a}{Intrinsic source radius (zero means point source), which is convolved with the
analysis PSF.}
\tablenotetext{b}{Statistical significance calculated according to \citet{article:LiMa:1983}.}
\tablenotetext{c}{99\% confidence level upper limit in units of $10^{-8}$ ph. m$^{-2}$ s$^{-1}$
calculated according to \citet{article:Rolke_etal:2005} above an energy threshold of 220 GeV, with
corresponding fluxes in percent of the steady Crab Nebula flux in parenthesis.}
\label{table:results}
\end{deluxetable}

\begin{deluxetable}{lccc}
\tablecolumns{4}
\tablewidth{0pc}
\tablecaption{\emph{Fermi}-LAT upper limits for the Coma cluster core}
\tablehead{
                      & $1-3$ GeV & $3-10$ GeV & $10-30$ GeV\\
\small{Spatial Model} & \small{Flux U.L.\tablenotemark{a}} (\small{Significance}) & \small{Flux
U.L.\tablenotemark{a}} (\small{Significance}) & \small{Flux U.L.\tablenotemark{a}}
(\small{Significance})\\
\hline
\multicolumn{4}{c}{\small{Spectral index $\alpha=2.1$}}
}
\startdata
\small{Point Source} & 2.745 (0.302) & 0.965 (0.000) & 1.002 (1.313)\\
\small{Disk: r = 0.2$^\circ$} & 3.066 (0.648) & 1.193 (0.218) & 1.165 (2.031)\\
\small{Disk: r = 0.4$^\circ$} & 3.515 (0.768) & 1.297 (0.177) & 1.234 (2.349)\\
\cutinhead{\small{Spectral index $\alpha=2.3$}}
\small{Point Source} & 2.803 (0.317) & 0.992 (0.000) & 1.006 (1.371)\\
\small{Disk: r = 0.2$^\circ$} & 3.130 (0.654) & 1.227 (0.286) & 1.166 (2.072)\\
\small{Disk: r = 0.4$^\circ$} & 3.586 (0.787) & 1.341 (0.273) & 1.236 (2.380)\\
\cutinhead{\small{Spectral index $\alpha=2.5$}}
\small{Point Source} & 2.854 (0.333) & 1.016 (0.000) & 1.009 (1.424)\\
\small{Disk: r = 0.2$^\circ$} & 3.184 (0.658) & 1.259 (0.349) & 1.167 (2.112)\\
\small{Disk: r = 0.4$^\circ$} & 3.647 (0.804) & 1.381 (0.362) & 1.237 (2.408)\\
\enddata
\tablenotetext{a}{99\% confidence level upper limit in units of $10^{-6}$ ph. m$^{-2}$ s$^{-1}$.}
\label{table:fermi}
\end{deluxetable}

\del{
\begin{deluxetable}{lcccccc}
\tablecolumns{7}
\tablewidth{0pc}
\tablecaption{\emph{Fermi}-LAT upper limits for the Coma cluster core}
\tablehead{
\colhead{Spatial model} & 
\multicolumn{3}{c}{S[$\sigma$]\tablenotemark{a}} & 
\multicolumn{3}{c}{Flux U.L.\tablenotemark{b}} \\
\colhead{} & 
\colhead{1-3 GeV} & 
\colhead{3-10 GeV} & 
\colhead{10-30 GeV} & 
\colhead{1-3 GeV} & 
\colhead{3-10 GeV} & 
\colhead{10-30 GeV} \\
\hline
\multicolumn{7}{c}{Spectral index $\alpha=2.1$ (fixed)}
}
\startdata
Point source & 0.30 & 0 & 1.3 & 1.73 & 4.87 & 0.627 \\
Disk, $r=0.2^{\circ}$ & 0.65 & 0.22 & 2.0 & 2.03 & 0.691 & 0.783 \\
Disk, $r=0.4^{\circ}$ & 0.77 & 0.18 & 2.3 & 2.39 & 0.757 & 0.847 \\
\cutinhead{Spectral index $\alpha=2.5$ (fixed)}
Point source & 0.33 & 0 & 1.42 & 1.81 & 0.526 & 0.635 \\
Disk, $r=0.2^{\circ}$ & 0.66 & 0.35 & 2.1 & 2.12 & 0.748 & 0.785 \\
Disk, $r=0.4^{\circ}$ & 0.80 & 0.36 & 2.4 & 2.50 & 0.833 & 0.849 \\
\enddata
\tablenotetext{a}{Statistical significance.}
\tablenotetext{b}{95\% confidence level upper limit in units of $10^{-6}$ ph. m$^{-2}$ s$^{-1}$.}
\label{table:fermi}
\end{deluxetable}
}

\begin{deluxetable}{cccccc}
\tablecolumns{6}
\tablewidth{0pc}
\tablecaption{Constraints on the relative CR pressure in the Coma cluster core (simplified analytic
  model) for different spatial extensions and predicted fluxes for the energy bands 1-3 GeV and
  $>220$~GeV (simulation-based model).} \rotate
\tabletypesize{\small}
\tablehead{
\colhead{RoI\tablenotemark{a} [deg]} 
    & \multicolumn{3}{c}{analytic model: $X_{\CR,\,\rmn{max}}~[\%]$\tablenotemark{b} }
    & \colhead{$F_\rmn{sim}(E)$\tablenotemark{c}} & \colhead{$F_\rmn{sim}(>E)$\tablenotemark{d}}   \\
& \colhead{$\alpha=2.1$} & \colhead{$\alpha=2.3$} & \colhead{$\alpha=2.5$} & & 
}
\startdata
\multicolumn{6}{c}{VERITAS constraints:}\\
\hline
0.13 & 10  & 23 & 97 & 0.52  & 1.9  \\
0.2  & 4.8 & 10 & 43 & 0.80  & 2.9  \\
0.4  & 6.7 & 15 & 62 & 1.21  & 4.4  \\
\cutinhead{{\em Fermi} constraints:}
0.13 & 5.2 & 3.5 & 4.7 & 2.2  & 1.4  \\
0.2  & 3.5 & 2.4 & 3.1 & 3.4  & 2.1  \\
0.4  & 2.5 & 1.7 & 2.3 & 5.2  & 3.2  \\
\enddata
\tablenotetext{a}{Intrinsic source radius; we approximate the aperture of the point source-limits by the
analysis PSF of 0.13$^\circ$.}
\tablenotetext{b}{Constraint on the relative CR pressure,
$X_{\CR} = \expval{P_\CR}/ \expval{P_\rmn{th}}$, which was assumed to be constant through the
cluster and calculated according to \citet{article:PfrommerEnsslin:2004b}.}
\tablenotetext{c}{Differential gamma-ray flux $E^{2}dF/dE$ from the simulation-based analytic model
by \citet{article:PinzkePfrommer:2010}: at $E=220$ GeV in units of $10^{-6}$ GeV m$^{-2}$ s$^{-1}$
for VERITAS and between $E=1-3$ GeV in units of $10^{-9}$ GeV m$^{-2}$ s$^{-1}$ for Fermi. }
\tablenotetext{d}{Integrated gamma-ray flux from the simulation-based analytic model by
\citet{article:PinzkePfrommer:2010}: above $E=220~\rmn{GeV}$ in units of $10^{-9}$ ph. m$^{-2}$
s$^{-1}$ for VERITAS and  above $E=1~\rmn{GeV}$ in units of $10^{-6}$ ph. m$^{-2}$ s$^{-1}$ for
Fermi.} 
\label{table:constraints_simple}
\end{deluxetable}



\begin{deluxetable}{ccccccc}
\tablecolumns{7}
\tablewidth{0pc}
\tablecaption{Minimum gamma-ray fluxes in the hadronic model of radio halos, where the
radio-emitting electrons are secondaries from CR interactions and corresponding minimum relative
CR pressures for Coma} \rotate
\tabletypesize{\small}
\tablehead{
\colhead{RoI\tablenotemark{a} [deg]} 
& \multicolumn{3}{c}{$F_{\gamma,\,\rmn{min}}(>E)$\tablenotemark{b}} 
& \multicolumn{3}{c}{$X_{\CR,\,\rmn{min}}~[\%]$\tablenotemark{c} } \\
& \colhead{$\alpha=2.1$} & \colhead{$\alpha=2.3$} & \colhead{$\alpha=2.5$}
& \colhead{$\alpha=2.1$} & \colhead{$\alpha=2.3$} & \colhead{$\alpha=2.5$}
}
\startdata
\multicolumn{7}{c}{VERITAS energy range, $E>220$~GeV:}\\
\hline
0.13 & 1.6 & 0.7 & 0.3 & 0.067 & 0.061 & 0.11 \\
0.2  & 3.1 & 1.4 & 0.6 & 0.078 & 0.072 & 0.13 \\
0.4  & 6.3 & 2.8 & 1.3 & 0.098 & 0.090 & 0.16 \\
\cutinhead{{\em Fermi} energy range, 1--3~GeV:}
0.13 & ~~3.5 & ~~4.8 & ~~6.4 & 0.067 & 0.061 & 0.11 \\
0.2  & ~~6.8 & ~~9.3 &  12.5 & 0.078 & 0.072 & 0.13 \\
0.4  &  13.5 & 18.6  &  25.0 & 0.098 & 0.090 & 0.16 \\
\enddata
\tablenotetext{a}{Intrinsic source radius; we approximate the aperture of the point source-limits by the
analysis PSF of 0.13$^\circ$.}
\tablenotetext{b}{Minimum gamma-ray flux of the hadronic model described in Sect~\ref{sec:Fmin}. 
Values are in units of $10^{-10}$ ph. m$^{-2}$ s$^{-1}$ (VERITAS energy range) and $10^{-8}$ ph. m$^{-2}$
s$^{-1}$ {\em (Fermi) energy range}.}
\tablenotetext{c}{Minimum relative CR pressure, $X_{\CR,\,\rmn{min}}$, in the
hadronic model described in Sect~\ref{sec:Fmin}.}
\label{table:constraints}
\end{deluxetable}


\begin{deluxetable}{ccccccc}
\tablecolumns{4}
\tablewidth{0pc}
\tablecaption{Constraints on magnetic fields in the hadronic model of the Coma radio halo and the
corresponding relative CR pressure (at the largest emission radius) such that the model reproduces the
observed radio surface brightness profile.}
\tablehead{
 &
\multicolumn{6}{c}{Minimum magnetic field, $B_{0,\rmn{min}}~[\mu\rmn{G}]$:}\\
 & \multicolumn{3}{c}{VERITAS constraints:} & \multicolumn{3}{c}{{\em Fermi} constraints:} \\
\colhead{$\alpha_{B}$}
 & \colhead{$\alpha=2.1$} & \colhead{$\alpha=2.3$} & \colhead{$\alpha=2.5$}
 & \colhead{$\alpha=2.1$} & \colhead{$\alpha=2.3$} & \colhead{$\alpha=2.5$}
}
\startdata
0.3 & 0.69 & 0.57 & 0.48 & 1.14 & 1.63 & 2.25 \\
0.5 & 0.97 & 0.80 & 0.68 & 1.60 & 2.28 & 3.17 \\
0.7 & 1.40 & 1.17 & 0.99 & 2.32 & 3.32 & 4.61 \\
\hline
& \multicolumn{6}{c}{Corresponding $X_{\CR,\rmn{max}}\, (1\,\rmn{Mpc})$:}\\ 
\hline
0.3 & 0.46 & 1.05 & ~~4.55 & 0.17 & 0.11 & 0.15 \\
0.5 & 0.74 & 1.70 & ~~7.47 & 0.27 & 0.18 & 0.25 \\
0.7 & 1.09 & 2.59 &  11.55 & 0.41 & 0.28 & 0.37 \\
\enddata
\tablecomments{The parameters of the magnetic field are the magnetic decline, $\alpha_B$, and the
central field strength, $B_{0}$, which are defined by
$B(r) = B_{0}\,[n_\rmn{e}(r)/n_{\rmn{e}}(0)]^{\alpha_B}$.}
\label{table:Bmin}
\end{deluxetable}


\begin{deluxetable}{ccc}
\tablecolumns{3}
\tablewidth{0pc}
\tablecaption{Astrophysical factors}
\tablehead{
\colhead{RoI [deg]} &
\colhead{$\expval{J}_{\rmn{signal}}$ [GeV$^{2}$ cm$^{-5}$ sr]} & 
\colhead{$\alpha\expval{J}_{\rmn{bkg}}$ [GeV$^{2}$ cm$^{-5}$ sr]}
}
\startdata
0 & $5.7\times 10^{16}$ & $1.3\times 10^{14}$ (negligible) \\
0.2 & $8.1\times 10^{16}$ & $4.4\times 10^{14}$ ($<0.01\expval{J}_{\rmn{signal}}$, negligible) \\
0.4 & $9.4\times 10^{16}$ & $1.3\times 10^{15}$ ($\simeq0.01\expval{J}_{\rmn{signal}}$, negligible)
\enddata
\tablecomments{$\expval{J}_{\rmn{bkg}}$ is the astrophysical factor calculated for the background
region (ring method) and is used to estimate the level of gamma-ray contamination from DM.}
\label{table:astrofactor}
\end{deluxetable}

\begin{deluxetable}{lccc}
\tablecolumns{4}
\tablewidth{0pc}
\tablecaption{Limits on the DM annihilation cross section $\expval{\sigma v}$}
\tablehead{
\colhead{Channel} & RoI [deg] & $m_{\chi}$ [GeV] & $\expval{\sigma v}$ [cm$^{3}$ s$^{-1}$]
}
\startdata
W$^{+}$W$^{-}$ & 0 & 2000 & $1.1\times 10^{-20}$ \\
& 0.2 & 1900 & $4.3\times 10^{-21}$ \\
& 0.4 & 1900 & $8.4\times 10^{-21}$ \\
$b\bar{b}$ & 0 & 3500 & $1.2\times 10^{-20}$ \\
& 0.2 & 3400 & $4.4\times 10^{-21}$ \\
& 0.4 & 3500 & $8.7\times 10^{-21}$ \\
$\tau^{+}\tau^{-}$ & 0 & 670 & $2.4\times 10^{-21}$ \\
& 0.2 & 650 & $9.1\times 10^{-22}$ \\
& 0.4 & 660 & $1.8\times 10^{-21}$
\enddata
\tablecomments{Upper limits are for different integration regions and DM particle mass $m_{\chi}$
derived from the VERITAS gamma-ray flux upper limits presented in this work.}
\label{table:DMlimits}
\end{deluxetable}

\end{document}

