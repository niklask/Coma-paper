\documentclass[12pt,manuscript]{aastex}
%%\documentclass[twocolumn]{emulateapj}

\def\del#1{{}}
%\def\del#1{{\bf (DELETED TEXT)}}
\def\C#1{#1}
%\def\C#1{{\bf #1}}

\newcommand{\vdag}{(v)^\dagger}
\newcommand{\myemail}{nkarlsson@physics.umn.edu}
\newcommand{\expval}[1]{\left\langle #1 \right\rangle}
\newcommand{\RA}[3]{#1$^{\mathrm{h}}$#2$^{\mathrm{m}}$#3$^{\mathrm{s}}$}
\newcommand{\Dec}[3]{#1$^{\circ}$#2\arcmin#3\arcsec}
\newcommand{\rmn}{\mathrm}
\newcommand{\CR}{\mathrm{CR}}
\newcommand{\dps}{\displaystyle}
\newcommand{\eps}{\varepsilon}
\newcommand{\bra}{\langle}
\newcommand{\ket}{\rangle}

\shorttitle{Coma Cluster Observations with VERITAS}
\shortauthors{VERITAS collaboration}

\begin{document}

%\title{VHE Gamma-Ray Observations of the Coma Cluster of Galaxies with VERITAS}
\title{Constraints on Cosmic-Ray Particle Populations, Magnetic Fields, and Dark Matter from VHE Gamma-Ray Observations of the Coma Cluster of Galaxies with VERITAS}
\author{N. Karlsson\altaffilmark{1,2}, M. Pohl\altaffilmark{3,4}}
\affil{(the VERITAS collaboration)} 
%%\email{nkarlsson@physics.umn.edu}
\and
\author{S. Federici\altaffilmark{3,4}, C. Pfrommer\altaffilmark{1,5,6},
A. Pinzke\altaffilmark{7}}

\altaffiltext{1}{Corresponding authors: N. Karlsson, nkarlsson@physics.umn.edu \& C. Pfrommer, christoph.pfrommer@h-its.org}
\altaffiltext{2}{School of Physics and Astronomy, University of Minnesota, Minneapolis, MN 55455, USA}
\altaffiltext{3}{DESY, Platanenallee 6, 15738 Zeuthen, Germany}
\altaffiltext{4}{Institut f\"ur Physik und Astronomie, Universit\"at Potsdam, 14476 Potsdam-Golm, Germany}
\altaffiltext{5}{Heidelberg Institute	for Theoretical	Studies, Schloss-Wolfsbrunnenweg 35, D-69118 Heidelberg, Germany}
\altaffiltext{6}{Canadian Institute for Theoretical Astrophysics, University of Toronto, Toronto, ON, M5S 3H8, Canada}
\altaffiltext{7}{Department of Physics, University of California, Santa Barbara, CA 93106, USA}

%%\journalinfo{The Astrophysical Journal}
%%\submitted{preprint}

\begin{abstract}
Theoretical models of galaxy clusters suggest they may be sources of very high-energy (VHE) gamma-ray emission. Observations of radio halos and relics in clusters indicate populations of relativistic electrons, which are either injected into the intra-cluster medium and accelerated by turbulent shocks or secondaries from interactions of cosmic-ray hadrons (protons and ions) with the intra-cluster gas. Those interactions would also produce gamma rays through the decay of neutral pions. Reported here are observations of the Coma cluster of galaxies with the VERITAS array of imaging Cherenkov telescopes. The observations were made between March and May 2008. Radio observations at 1.41 GHz taken with the Green Bank Telescope and X-ray observations with the ROSAT All-Sky survey was used to a priori select regions of interest in the field of view. No significant VHE gamma-ray emission from the Coma cluster was detected by VERITAS. Several emission regions are considered, including the cluster core as both unresolved (point-like) and spatially extended, and a select few cluster members. The flux upper limits at the 99\% confidence level measured are on the order of $2-5\times 10^{-8}$ ph. m$^{-2}$ s$^{-1}$, which is X times higher than the model-predicted minimum gamma-ray flux.

The gamma-ray flux upper limits have been used to constrain the populations of cosmic-ray particles and magnetic fields in the cluster. The relative cosmic-ray pressure $X_{\mathrm{CR}}$ is less than\ldots. The observations imply a lower limit on the intra-cluster magnetic field on the order of a $\mu$G depending on the magnetic decline and gamma-ray spectral index. Finally, a constraint on the thermally averaged product of the total self-annihilation cross section and the relative velocity of the dark-matter particles is derived, $\expval{\sigma v}\lesssim 10^{-21}$ cm$^{3}$ s$^{-1}$ for $m_{\chi}>X$ GeV/c$^{2}$.
\end{abstract}

%\keywords{cosmic rays: interactions --- galaxies --- galaxy clusters -- VHE}
\keywords{cosmic rays --- gamma rays: observations --- gamma rays: VHE --- galaxies: clusters: general --- galaxies: clusters: individual: Coma (ACO 1656) --- dark matter --- magnetic fields}

\section{Introduction}
Clusters of galaxies are the largest observable structures in the Universe, with typical diameters of several Mpc and masses on the order of $10^{14}$ to $10^{15} M_{\odot}$. These gravitationally bound objects have formed on increasingly larger scales during the evolution of the Universe through the process of cosmic structure formation \citep{article:Voit:2005}. Although most of the cluster mass comes from large amounts of dark matter (DM) embedded in a cluster, as indicated by cluster dynamics and weak gravitational lensing \citep[][]{article:DiaferioSchindlerDolag:2008}, gas making up the intra-cluster medium (ICM) contributes about 15\% of the total cluster mass. Individual galaxies contribute only a small amount (about 5\%) to the cluster mass. The ICM gas mass also comprise a significant fraction of the observable (baryonic) matter in the Universe.

The ICM is a hot, $T\sim 10^{8}$ K, plasma emitting thermal bremsstrahlung radiation in the soft X-rays regime \citep[see, e.g.][]{article:Petrosian:2001}. Heating of this plasma is thought to come primarily from collisionless structure-formation shocks, due to accretion and merger processes. Such shocks and turbulences in the ICM gas and magnetic fields also provide a means to efficiently accelerate particles \citep[see, e.g.,][]{article:ColafrancescoBlasi:1998, article:Ryu_etal:2003}. Many clusters do indeed feature mega-parsec scale halos of non-thermal radio emission, indicative of a population of relativistic electrons and positrons (leptons) and magnetic fields permeating the ICM \citep{article:Cassano_etal:2010}. Recent observations of possible non-thermal emission from clusters in the the extreme ultraviolet \citep[EUV; ][]{article:SarazinLieu:1998} and hard X-rays \citep{article:RephaeliGruber:2002, article:Fusco-Femiano_etal:2004, article:Eckert_etal:2007} provide further indications of relativistic particle populations in clusters, although the interpretation of these observations have been disputed \citep[see, e.g.][]{article:Wik_etal:2009}.

Galaxy clusters have for many years been proposed as sources of gamma rays. If shock acceleration in the ICM is an efficent process, a population of highly relativistic cosmic-ray (CR) hadrons, in the form of protons and heavy ions, in the ICM is to be expected. The main energy-loss mechanism for CR hadrons at high energies is pion production through the interaction of CRs with nuclei in the ICM. Pions are short lived and decay; the charged pions to electrons and positrons,\footnote{A charged pion decays to a muon or antimuon, which subsequently decay to an electron or positron.} and the neutral pion to gamma rays. Due to the low density of the ICM ($n_{\mathrm{ICM}}\sim 10^{-3}$ cm$^{-3}$) and the sheer size of a cluster, CR hadrons will be confined in the cluster on very long timescales \citep[][]{article:Volk_etal:1996, article:Berezinsky_etal:1997} allowing them to accumulate over time. 

\paragraph{Rewrite this paragraph!}
Depending on the CR energy density, quantified by the fraction of energy in CRs to the thermal energy in a cluster, $\eta=\expval{ E_{\mathrm{CR}}/E_{\mathrm{th}}}$, an observable flux of high-energy gamma rays can be expected. This fraction could plausibly be as large as 50\% \citep{article:Ryu_etal:2003}, but a much smaller fraction of only a few percent is required in order to produce a gamma-ray flux observable with current generation gamma-ray telescopes. 

Gamma-ray emission can also be produced by Compton up-scattering of ambient photons, for example cosmic microwave background (CMB) photons, on ultra-relativistic leptons. Those leptons can either be secondaries from the above mentioned CR interactions or be injected into the ICM by powerful cluster members and be further accelerated by diffusive shock acceleration or turbulent reacceleration processes.

A third mechanism for gamma-ray production in a galaxy cluster is self-annihilation of a DM particle, e.g. a weakly interacting massive particle (WIMP). As already mentioned, about 80\% of the cluster mass is in the form of dark matter. Recent works \citep{article:EvansFerrerSarkar:2004, article:BergstromHooper:2006} have shown that the gamma-ray emissivity from DM annihilations is expected to be proportional to the DM density along the line of sight, making clusters of galaxies potentially good candidates for DM searches, despite their large distances. 

While several observations of clusters of galaxies have been made with satellite-borne and ground-based telescopes, a detection of gamma-ray emission from a cluster has yet to be made.  Observations with EGRET \citep{article:Sreekumar_etal:1996, article:Reimer_etal:2003} and the Fermi Gamma-ray Space Telescope \citep{article:Ackermann_etal:2010} have provided upper limits on the gamma-ray fluxes (typically $\sim10^{-9}$ ph cm$^{2}$ s$^{-1}$ for Fermi observations) for several galaxy clusters in the MeV to GeV band. Upper limits on the VHE gamma-ray flux from a few select clusters, including the Coma cluster, have been provided by observations with ground-based imaging atmospheric Cherenkov telescopes \citep[IACTs;][]{article:Perkins_etal:2006, inproc:Perkins_etal:2008, article:Aharonian_etal:2009, article:Aleksic_etal:2010}.

The Coma cluster of galaxies (ACO 1656) is one of the most thouroughly studied clusters over all wavelengths \citep{article:Voges_etal:1999}. Located at a distance of about 100 Mpc \citep[$z=0.023$;][]{article:StrubleRood:1999}, it is one of the closest and also one of the most massive clusters \citep[$M \sim 10^{15}M_{\odot}$;][]{article:Smith:1983, article:Kubo_etal:2008}. It hosts both a giant radio halo \citep{article:Giovannini_etal:1993,article:Thierbach_etal:2003} and peripheral radio relic, which appears connected to the radio halo with a ``diffuse'' bridge \citep[see discussion in][]{article:BrownRudnick:2010}. It has been suggested that the relic is an infall shock \citep{article:Ensslin_etal:1998} and succesively demonstrated by cosmological simulations that model the non-thermal emission processes \citep{article:PfrommerEnsslinSpringel:2008, article:Pfrommer:2008, article:Battaglia_etal:2009, article:Skillman_etal:2011}. Extended soft X-ray (SXR) emission is evident from the ROSAT all-sky survey in the 0.1 to 2.4 keV band \citep{article:BrielHenryBohringer:1992} and observations with XMM-Newton \citep{article:Briel_etal:2001} revealed substructure in the X-ray halo. As such it is a natural candidate for gamma-ray observations. 

In this article, results from the VERITAS observations of the Coma cluster of galaxies are reported, 
together with the resulting constraints on cosmic-ray particle populations, magnetic fields, and dark matter in the cluster. Throughout the analysis, a present day Hubble constant of $H_{0} = 100h$ km s$^{-1}$ Mpc$^{-1}$ with $h=0.7$ has been used.

\section{VERITAS Observations and Analysis}
The VERITAS gamma-ray detector \citep{article:Weekes_etal:2002} is an array of four 12 m diameter imaging atmospheric Cherenkov telescopes \citep[IACTs;][]{article:Holder_etal:2006} located at an altitude of 1268 m a.s.l. at the Fred Lawrence Whipple Observatory in southern Arizona (31$^{\circ}$~40\arcmin~30\arcsec~N, 110$^{\circ}$~57\arcmin~07\arcsec~W). Each of the telescopes is equipped with a 499 pixel camera covering a 3.5$^{\circ}$ field of view. The array, completed in the spring of 2007, is designed to detect gamma-ray emission from astrophysical objects in the energy range from 100 GeV to more than 30 TeV. The energy resolution is about 15\% and the angular resolution (68\% containment) is about 0.1$^{\circ}$ per event at 1 TeV. The sensitivity of the array allows for detection of a point source with a flux of 1\% of the steady Crab Nebula flux above 300 GeV at the $5\sigma$ level in under 30 hours.\footnote{The integral flux sensitivity above 300 GeV was improved by about 30\% with the move of one telescope in the summer of 2009. The move had no impact on the this data set.}

The Coma cluster was observed with VERITAS between March and May in 2008 with all four telescopes fully operational. The total exposure amounts to 18.6 hours of quality-selected live time, i.e. time periods of astronomical darkness, clear sky conditions and without technical problems with the array. The center of the cluster was tracked in \emph{wobble} mode, where the expected source location is offset from the center of the field of view, to allow for simultaneous background estimation \citep{article:Formin_etal:1994}. All of the observations were made in a small range of zenith angles averaging $\sim 21^{\circ}$.

The data analysis was performed following the standard VERITAS procedures described in \citet{inproc:Cogan_etal:2007} and \citet{inproc:Daniel_etal:2007}. Prior to event reconstruction and selection, all shower images are calibrated and cleaned. Showers are then reconstructed for events with at least two contributing telescope images that pass the following quality selection criteria: more than four participating pixels in the camera, number of photoelectrons in the image is larger than 75, and the distance from the image centroid to the center of the camera is less than $1.43^{\circ}$. These quality selection criteria impose an energy threshold\footnote{Defined as the energy corresponding to the maximum of the product function of the observed spectrum with the collection area} of about 220 GeV. In addition, events for which only images from the two closest-spaced telescopes\footnote{In the array configuration prior to summer 2009, two telescopes had a separation of only 35 m.} survive quality selection are rejected, as they introduce an irreducible high background rate, due to local muons, degrading the instrument sensitivity \citep{article:MaierKnapp:2007}.

Gamma-ray like events are separated from the CR background by imposing cuts on scaled parameters \citep{article:Aharonian_etal:1997, article:Krawczynski_etal:2006} calculated from a parameterized moment analysis of the shower images \citep{inproc:Hillas:1985}. The gamma-hadron separation cuts used in this analysis were optimized a priori for a weak point source (3\% Crab Nebula flux level) and a photon index of -2.4 using data taken on the Crab Nebula during the same epoch and are termed \emph{standard} cuts. Because the VHE gamma-ray spectrum for the Coma cluster is expected to be hard, photon index of about -2.3, these standard cuts are suitable for the analysis of the Coma cluster data set. 

The Coma cluster is a very rich cluster of galaxies with many plausible sites for gamma-ray emission: the core region, the peripheral radio relic, as well as individual powerful cluster member galaxies. VERITAS has a large enough field of view to allow investigation of several of these scenarios. In this work the focus has been on the core region and three cluster members. The latter are treated as point sources in the analysis, whereas the core is treated as both a point source and a mildly extended source. The intrinsic source extension is assumed to be $0.2^{\circ}$ and $0.4^{\circ}$, based on the extension of the thermal soft X-ray emission from the core. There is evidence of a recent merger event between the two central galaxies NGC 4889 and NGC 4874 \citep{article:Tribble:1993}. There is also evidence for an excess of non-thermal X-ray emission from these galaxies as well as the galaxy NGC 4921 \citep{article:Neumann_etal:2003}. Therefore, searches for point-like VHE gamma-ray emission have been conducted at the locations of these galaxies. The regions of interest considered in this work are summarized in Table \ref{table:roi}.

The ring background model \citep{article:Aharonian_etal:2001} is used to estimate the background due to CRs misinterpreted as gamma rays (the cuts described above reject more than 99\% of all CRs). The total number of events in a given region of interest is then compared to the estimated background from the off-source region scaled by the ratio of the solid angles to produce a final excess. The VHE gamma-ray significance is then calculated using the method by \citet{article:LiMa:1983}. Significance skymaps over the VERITAS field of view produced with a $0.2^{\circ}$ sampling radius are shown in Figure \ref{fig:skymaps} with overlaid X-ray and radio contours from the ROSAT all-sky survey \citep{article:BrielHenryBohringer:1992}  and GBT 1.4 GHz observations \citep{article:BrownRudnick:2010} respectively.

\section{VERITAS Results}
No significant excess of VHE gamma-rays from the Coma cluster was detected with VERITAS, as illustrated by the $\theta^{2}$ distribution shown in Figure \ref{fig:thetasq}. The $\theta^{2}$ distribution is a radial comparison of the ON-source event distribution with the estimated background. It extends out to a radius of $0.42^{\circ}$ to cover all regions of interest considered in this work. A 99\% confidence level upper limit is calculated for each region of interest using the method described by \citet{article:Rolke_etal:2005} assuming a Gaussian background. A lower bound of zero is imposed on the signal rate to handle the negative excesses seen for several of the regions of interest. Figure \ref{fig:sigdist} shows that the VERITAS data is well fit by a Gaussian with a mean close to zero and a standard deviation within a few percent of unity. 

Table \ref{table:results} lists the upper limits for the selected regions of interest tabulated in Table \ref{table:roi}. These upper-limit calculations depend on the gamma-ray spectrum, which in this work is assumed to be a power law with a spectral index $\alpha=2.1, 2.3$, and $2.5$.

\section{Fermi-LAT Analysis and Results}
The Large Area Telescope (LAT) on board the \emph{Fermi} Gamma-ray Space Telescope (\emph{Fermi}) has observed the Coma cluster in all-sky survey mode since its launch in June 2008. The \emph{Fermi}-LAT is sensitive to gamma rays in the 20 MeV to $\sim$300 GeV energy range and is thus very complimentary to the VERITAS observations. \citet{article:Ackermann_etal:2010} reported on the search for gamma-ray emission from thirty-three galaxy clusters in the data from the first 18 months, including the Coma cluster, for which an upper limit of $4.58\times 10^{-9}$ ph. cm$^{-2}$ s$^{-1}$ in the 0.2 to 100 GeV energy band was reported. This limit is expected to improve as the exposure is increased. In this work an updated analysis which includes additional data taken during the period 2010 February 5 to 2011 June 10 is presented as a compliment to the VERITAS results.

The analysis of this work follows the same procedure as described in detail in \citet{article:Abdo_etal:2009}. To eliminate albedo gamma rays from the Earth limb, a zenith angle cut of 105$^{\circ}$ is applied. The \emph{Fermi} Science Tools version v9r23p1 and Instrumental Response Functions (IRFs) P6\_V11\_Diffuse (a model of the spatial distribution of photon events calibrated  in flight) were used throughout this work. Only photons above 1 GeV are considered, where the constraints on the flux are most meaningful. The \emph{Fermi}-LAT collaboration estimates the systematic uncertainties at 10 GeV to be around 20\%.\footnote{\url{http://fermi.gsfc.nasa.gov/ssc/data/analysis/LAT\_caveats.html}} Flux upper limits at the 95\% confidence level in the energy range 1-100 GeV, assuming both a unresolved, point-like or spatially extended emission, are tabulated in Table \ref{table:fermi}. The gamma-ray emission is assumed to follow a power-law distribution with a photon index $\alpha=2.1$ and $2.5$.

\section{Gamma Ray Emission from Cosmic Rays}
We decided to adopt a multi-faceted approach to constrain the CR-to-thermal pressure distribution in the Coma cluster with the VERITAS and \emph{Fermi}-LAT upper limits from this work. This approach includes (1) an analytic model derived from cosmological hydrodynamical simulations of the formation of galaxy clusters, (2) a simplified multi-frequency analytical approach that assumes a constant CR-to-thermal energy density and a power-law spectrum in momentum, and (3) an approach that uses the observed luminosity of the giant radio halo in Coma to place a lower limit on the expected $\gamma$-ray flux in the hadronic model -- independent of the magnetic field distribution. This translates into a minimum CR pressure which, if challenged by tight gamma-ray limits/detections, enables scrutinizing the hadronic interaction model of the formation of giant radio halos. Modelling the CR distribution through different techniques enable to bracket our ignorance about the underlying plasma
physics that shape the CR distribution and hence reflect the Bayesian priors that are imposed on the modelling \citep[see][for a discussion]{article:PinzkePfrommerBergstrom}.

\subsection{Simulation-based approach}
We adopt the universal spectral and spatial gamma-ray model developed by \citet{article:PinzkePfrommer:2010} to estimate the emission from decaying neutral pions that dominates over the IC emission from primary and secondary electrons above 100 MeV in clusters. Given a density profile as, e.g., inferred by cosmological simulations or X-ray observations, the analytic approach models the CR distribution and the associated radiative emission processes from radio to the gamma-ray band. This formalism was derived from high resolution simulations of clusters of galaxies that included radiative hydrodynamics, star formation, supernova feedback, and followed CR physics by tracing the most important injection and loss processes self-consistently while accounting for the CR pressure in the equation of motion \citep{article:PfrommerSpringelEnsslinJubelgas, article:EnsslinPfrommerSpringelJubelgas:2007, article:JubelgasSpringelEnsslinPfrommer:2008}. The results are in line with earlier numerical results on some of the overall characteristics of the CR distribution and the associated radiative emission processes \citep{article:DolagEnsslin, article:MiniatiRyuKangJones:2001, article:Miniati:2003, article:Pfrommer_etal:2007, article:PfrommerEnsslinSpringel:2008, article:Pfrommer:2008}.
 
We note that the overall normalization of the CR and gamma-ray distribution scales with the maximum acceleration efficiency at structure formation shock waves. Following recent observations at supernova remnants \citep{article:Helder_etal:2009} as well as theoretical studies \citep{article:KangJones:2005}, we adopt an optimistic but nevertheless realistic value of this parameter and assume that 50\% of the dissipated energy at strong shocks is injected into CRs while this efficiency rapidly decreases for weaker shocks. Since the vast majority of internal formation shocks (merger and flow shocks) are weak shocks with Mach numbers $M\lesssim3$ \citep[e.g.,][]{article:Ryu_etal:2003}, they do not significantly contribute to the CR population in clusters. Instead, strong shocks during the formation epoch of clusters and strong accretion shocks at the present time (at the boundary of voids and filaments/super-cluster regions) dominate the injection of CRs which are adiabatically transported through the cluster. Hence, these model predictions provide a plausible upper limit for the CR contribution from structure formation shocks in galaxy clusters which can be scaled with the effective acceleration efficiency. Other possible CR sources, such as AGN and starburst-driven galactic winds have been neglected for simplicity which could in principle increase the expected gamma-ray yield.

There cosmological simulations only considered advective transport of CRs by turbulent gas motions which produces centrally enhanced profiles.  However, other means of CR transport such as diffusion and streaming may flatten the CR radial profiles. It has been suggested that advection is dominant only during mergers \citep{article:EnsslinPfrommerMiniatiSubramanian:2011}, in which case relaxed clusters could have a reduced gamma-ray luminosity by up to a factor of five compared with merging clusters. The goal of this and other future work of seeking for gamma-ray emission from clusters is to constrain the acceleration physics and transport properties of CRs.

We adopt the electron density profile for the Coma cluster that has been inferred from ROSAT X-ray observations \citep{article:BrielHenryBohringer:1992} and model the spatial and spectral distribution of CRs according to the model by \citet{article:PinzkePfrommer:2010} that neglects the contribution of cluster galaxies. Figure \ref{fig:spectrum} shows the expected spectrum of Coma, normalized to the total flux within the virial radius. [Add {\em Fermi} results]

Figure \ref{fig:xcr} shows the relative CR pressure, $X_{\CR} = \expval{P_{\CR}}/\expval{P_\rmn{th}}$ as a function of radial distance from the Coma cluster center (dashed lines). To compute the CR pressure, we assume a low-energy cutoff of the CR distribution at $T = 0.8\,m_\rmn{p} c^2$ reflecting the high cooling rates at low CR energies. The relative CR pressure rises toward the outer regions on account of the higher efficiency of CR acceleration at the peripheral strong accretion shocks compared to weak central flow shocks. Adiabatic compression of a mixture of CRs and thermal gas disfavours the CR pressure relative to the thermal pressure on account of the softer equation of state of CRs. The weak increase of $X_{\CR}$ toward the core is due to the comparably fast thermal cooling of gas which diminishes the thermal support relative to that in CRs.

Table \ref{table:constraints} shows the expected gamma-ray fluxes within the regions of interest. For the most constraining regions tested (within an aperture of radius 0.2$^{\circ}$), those are a factor of 7.2 below the inferred upper limits (see Table \ref{table:results} and assuming a spectral index of $\alpha=2.3$ which matches the simulated one at energies $E=200$ GeV). Hence we scale the averaged relative CR pressure %$X_{\CR}$
% = \bra P_\CR \ket /\bra P_\rmn{th} \ket$ 
of our model by that factor, keep the spatial behavior and obtain an integrated limit of the relative CR pressure of $X_{\CR}<0.091$ within 0.2$^{\circ}$ that translates to a limit within the cluster virial radius of $X_\CR<0.13$ (solid lines of Figure \ref{fig:xcr}).

\subsection{Simplified analytical model}
To complement the simulation-inspired analysis and to facilitate comparison with earlier analytical work, we repeat the data analysis with a simplified analytical model that assumes a power-law CR spectrum and a constant relative CR pressure, i.e., we adopt the {\em isobaric model of CRs}. Since a priori, the CR spectral index is unconstrained, we vary it within plausible ranges of $2.1<\alpha_{\rmn{p}}<2.5$. This spectral range is compatible with the radio spectral index of the giant radio halo of Coma after accounting for the spectral
steepening at frequencies $\nu\sim5~\rmn{GHz}$ due to the Sunyaev-Zel'dovich effect \citep{article:PfrommerEnsslin:2004b}\footnote{Assuming a magnetic field of 1 $\mu$G, the CR protons responsible for the GHz radio emitting electrons have an energy of $\sim100$ GeV and are $\sim$ 20 times less energetic than those CR protons responsible for 200-GeV gamma-ray emission.}. Table \ref{table:constraints} shows the resulting constraints on $X_\CR$ which range between 0.05 and 1, strongly
depending on the assumed spectral index.

[Add {\em Fermi} results, critical spectral index]

%
% This part was in \del{...} 
%
%the limit for 0.2 deg and alpha_p = 2.3 is compatible with the simulation based
%one, X_CR < 0.1. However, there is currently a discrepancy in our assumptions
%about the lower energy cutoff of CRs: in the simulations we use P/mp c = 0.8 (as
%simulated) while I set this to 0 in my simplified (isobaric) model. The
%discrepancy can be understood as follows:
%*) in the isobaric model, we would have to normalize our $X_\CR$ by a factor of 2/3
%for an effective spectral index of ~ 2.4 (between GeV and 200 GeV) as inferred
%from the simulations \citep[see Fig 1 in][]{article:PfrommerEnsslin:2004}).
%*) In the simulation model, we have an increased pressure by this factor of 3/2
%(in comparison to a power law spectrum with alpha_TeV ~ 2.3 down to a cutoff in
%the normalized momentum of 0.8) due to the concave curvature of our spectrum.
%}

\subsection{Minimum gamma-ray flux}
For clusters that host radio halos we are able to derive a minimum gamma-ray flux in the hadronic model of CR interactions. The idea is based on the fact that a stationary distribution of CR electrons loses all its energy to synchrotron radiation for strong magnetic fields ($B \gg B_{\rmn{CMB}} \simeq 3.2 \mu$G) so that the ratio of gamma-ray to synchrotron flux becomes independent of the spatial distribution of CRs and thermal gas \citep{article:Voelk:1989, article:Pohl:1994}, in particular with $\alpha_{\nu}\simeq 1$ as the observed synchrotron spectral index.  Hence we can derive a minimum gamma-ray flux in the hadronic
model
\begin{equation}
\label{eq:Fmin}
\mathcal{F}_{\gamma,\rmn{min}} = \frac{\dps A_{\gamma}}{\dps A_{\nu}}\frac{\dps L_{\nu}}{\dps 4\pi D_{\rmn{lum}}^{2}},
\end{equation}
where $L_{\nu}$ is the observed luminosity of the radio mini-halo and $D_{\rmn{lum}}$ denotes the luminosity distance to the respective cluster. Lowering the magnetic field would require an increase in the energy density of CR electrons to reproduce the observed synchrotron luminosity and thus increase the associated gamma-ray flux. 

To derive a minimum gamma-ray flux that is comparable to the upper limits, we need to determined the radio flux within the corresponding angular regions. To this end, we fit the point-source subtracted, azimuthally averaged radio halo profile at 1.38 GHz \citep{article:Deiss_etal:1997} with a $\beta$-model,
\begin{equation}
\label{beta}
 S_{\nu} (r_{\bot})= S_{0} \left[ 1 + \left( \frac{r_{\bot}}{r_{\rmn{c}}}\right)^{2}\right]^{-3\beta + 1/2},
\end{equation}
where $S_{0} = 1.1 \times 10^{-3}\,\rmn{Jy\,arcmin}^{-2}$, $r_{\rmn{c}} = 450$ kpc, and $\beta = 0.78$.  Within the error bars, this profile is consistent with 326-MHz data taken by \citet{article:Govoni_etal:2001} when scaled with a radio spectral index of 1.15. 

The results for the minimum gamma-ray flux $F_{\rmn{min}}(>220~\rmn{GeV}$ and the minimum relative CR pressure, $X_{\CR,\,\rmn{min}} = F_{\rmn{min}}/F$ are shown in Table \ref{table:constraints}. Even in the most constraining case ($\alpha=2.1$, $<0.2^{\circ}$), these are a factor of about 60 below the upper limits obtained in this work. 

[Add {\em Fermi} results]

\section{Constraining the Magnetic Field}
\label{sec:B}
We have seen in the previous section that we can obtain an absolute lower limit on the gamma-ray emission in the hadronic model by assuming high magnetic fields, $B>B_\rmn{CMB}$. Hence, we can turn this around by using our upper limit on the gamma-ray emission (and by extension on the CR pressure) to infer a lower limit on the magnetic field that is needed to explain the observed radio emission which, however, assumes the validity of the hadronic model. Lowering the gamma-ray constraint will tighten (increase) the magnetic field limit. Once this gets in conflict with magnetic field measurements by means of other methods, e.g., Faraday rotation measure (RM)\footnote{Generally, Faraday RM analysis of the magnetic-field strength by, e.g., background sources through clusters are degenerate with the magnetic coherence scale and may be biased by the unknown correlation between magnetic and density fluctuations.}, this would challenge the hadronic model of radio halos. The method that we use to constrain the magnetic field inherits a dependence on the assumed spatial structure that we parametrize by
\begin{equation}
\label{eq:B}
B(r) = B_{0} \,\left(\frac{n_{\rmn{e}(r)}}{n_{\rmn{e}(0)}}\right)^{\alpha_B},
\end{equation}
as suggested by Faraday RM studies and numerical magneto-hydrodynamical (MHD) simulations \citep[][and references therein]{article:Bonafede_etal:2010, article:Bonafede_etal:2011}. In fact, the magnetic field in the Coma cluster is among the best constrained, because its proximity permits RM observations of
seven radio-sources located at projected distances of 50 to 1500 kpc from the cluster center. The best-fit model obtains $B_{0} = 4.7^{+0.7}_{-0.8}\,\mu$G and $\alpha_{B} = 0.5^{+0.2}_{-0.1}$ \citep{article:Bonafede_etal:2010}. We aim for constraining the central field strength, $B_{0}$, which depends on the magnetic decline, $\alpha_{B}$, that we vary (within reasonable ranges of $\Delta\alpha_{B}=0.2$ as suggested from the Faraday RM studies) to address this dependence. We proceed as follows:

\begin{enumerate}
\item Given a model for the magnetic field, we determine the profile of the relative CR pressure, $X_{\CR}$, so that the hadronically produced CR electrons exactly matches the observed radio halo emission over the entire extent. To this end, we de-project the fit to the surface brightness profile (using an Abel integral equation, see Appendix of \citealt{article:PfrommerEnsslin:2004b}) yielding the radio emissivity,
\begin{equation}
\label{eq:Coma:radio}
j_{\nu} (r) = \frac{S_{0}}{2\pi\, r_{\rmn{c}}}\,
\frac{6\beta - 1}{\left(1 + r^{2}/r_{\rmn{c}}^{2}\right)^{3 \beta}}\,
\mathcal{B}\left(\frac{1}{2}, 3\beta\right)
= j_{\nu,0} \left(1 + r^2/r_{\rmn{c}}^{2}\right)^{-3 \beta},
\end{equation}
where $\mathcal{B}$ denotes the beta function. Using ``isobaric CR model'' to compute the secondary radio emission profile uniquely determines $X_\CR(r)$. It is generically true for weak magnetic fields ($B<B_{\rmn{CMB}}$) in the outer parts of the Coma halo that the product $X_{\CR}(r)X_{B}(r)$ has to increase by a factor of about 100 toward the radio halo periphery to account for the observed extent. If we were to adopt a steeper magnetic decline such as $\alpha_{B}=0.5$ which produces a flat $X_{B}(r)$, the relative CR pressure would have to rise accordingly by a factor of 100.

\item Given this realization for $X_{\CR}$, we compute the pion-decay gamma-ray surface brightness profile, integrate the flux within a radius of $\{0.13, 0.2, 0.4\}$ degree, and scale the CR profile in order to match the corresponding VERITAS upper limits. This scaling factor, $X_{\CR,0}$, depends on the CR spectral index, $\alpha$, (assuming a power-law CR population for simplicity) and the radial decline of the magnetic field, $\alpha_{B}$.

\item In the last step, we scale the secondary radio emission profile linearly with $X_{\CR,0}$ and solve for $B_{0}$ subject to the boundary condition to match the observed synchrotron profile. Note that for $B_{0} \gg B_{\rmn{CMB}}$ and radio spectral index, $\alpha_{\nu}$, the solution would be degenerate.
% according to Eqn. (\ref{eq:Lnu}). 
\end{enumerate}

Table~\ref{table:Bmin} shows the results on the lower limit of the central magnetic field that range from $B_{0} = 0.4$ to $1.2\,\mu$G. There are a few interesting remarks in order. (1) The hardest CR spectral indices are most constraining because of the tightest upper limits. (2) For a steeper magnetic decline (larger $\alpha_{B}$), the CR number density needs to be larger to match the observed radio emission profiles which would yield a higher gamma-ray flux so that the upper limits are more constraining. This implies tighter lower limits for $B_{0}$. However, the largest choice of $\alpha_B=0.7$, which represents a flux-freezing magnetic field, is possibly too steep in the outer parts as it neglects the contributions from turbulent dynamos in these dynamically active regions. (3) Interestingly, in all cases, we use the 0.4$^{\circ}$ limit that is always most constraining for our choice of magnetic field configurations.  This is because we need a substantially increasing relative CR pressure profile to match the observed radio profiles (for a given magnetic realization); hence that CR realization produces a larger flux within 0.4$^{\circ}$ in comparison to the ``isobaric CR model'' assumption ($X_{\CR} = \rmn{const.}$) for which the 0.2$^{\circ}$ upper limits are more constraining; hence the constraints on $B_{0}$ get better for the larger integration area. Physically, this may be achieved by considering CR transport processes such as streaming which produces an increasing $X_{\CR}$-profile due to the larger available phase space in the outer cluster regions.

As a final word, we note that the $X_{\CR}$ profiles always obey the energy condition, i.e., $P_{\CR} < P_{\mathrm{th}}$ over the entire range of the radio halo emission ($< 1$ Mpc) when assuming a central magnetic field of $B_{0}=3\,\mu$G that is at the lower 2-$\sigma$ interval bound inferred from Faraday RM
\citep{article:Bonafede_etal:2010}. Typical values at 1 Mpc range from $X_{\CR} = 0.02$
($\alpha_{B}=0.3$, $\alpha=2.1$) to $X_{\CR} = 1$ ($\alpha_{B}=0.7$,
$\alpha=2.5$).\footnote{See Figure 3 in \citet{article:PfrommerEnsslin:2004a} for the entire parameter range assuming minimum energy conditions and \citet{article:PfrommerEnsslin:2004b}, Figure 7 for a parametrization as adopted in this study.} We caution, however, that the minimum-energy condition is violated at the outer radio halo boundary for the range of minimum magnetic-field values inferred by this study.

\section{Emission from Dark Matter Annihilations}


\section{Discussion}
VERITAS observed the Coma cluster of galaxies for a total of 18.6 hours of high-quality live time between March and May in 2008. 

Various approaches to model CR population and magnetic field distribution are complementary (so far just ideas, need to be formulated):
\begin{itemize}
\item simplified ``isobaric CR model'': simple and widely used in the literature. A priori, physically not justified, old fashioned, should be replaced by the approached below.
\item simulation-based analytical approach: ``first principle approach'' that predicts the CR and B distribution for a given set of assumptions. Powerful since it only required the density profile as input, but limited in the modeled physics (no active CR transport) that appears to be necessary to explain the radio halo bimodality.
\item pragmatic approach as employed in \S \ref{sec:B}: model CR and B to reproduce observed radio emission profile. Minimum energy conditions (or equivalently equipartition arguments) represent a special case within this approach. A priori, also physically not justified, but quite powerful as it shows what the ``correct'' model has to achieve and can point into the direction what the relevant physics has to deliever. Compare CR transport vs. simulation inspired models
\end{itemize}

\acknowledgments
This research was supported by grants from the U.S. Department of Energy, the U.S. National Science Foundation, and the Smithsonian Institution, by NSERC in Canada, by Science Foundation Ireland, and by STFC in the UK. We acknowledge the excellent work of the technical support staff at the FLWO and the collaborating institutions in the construction and operation of the instrument.

%%Facilities: \facility{VERITAS}.

\bibliographystyle{apj}
\bibliography{refs}

\begin{figure*}
\begin{center}
\scalebox{0.45}{\plotone{f1a.eps}}
\scalebox{0.45}{\plotone{f1b.eps}}
\end{center}
\caption{\emph{Left}: Smoothed significance map of the Coma cluster calculated from the observed excess VHE gamma-ray events over a $4.5^{\circ}\times 4.5^{\circ}$ field of view. The excess counts were derived using a ring background model \citep{article:Aharonian_etal:2001}. White contours show the X-ray counts per second in the 0.1 to 2.4 keV energy band from the ROSAT all-sky survey \citep{article:BrielHenryBohringer:1992}. \emph{Right}: Same as above but with overlaid contours from GBT radio observations at 1.4 GHz \citep{article:BrownRudnick:2010} where strong point sources have been subtracted. Shown are also the $0.2^{\circ}$ and $0.4^{\circ}$ radii (dashed cyan) considered for the extended source analyses presented here.}
\label{fig:skymaps}
\end{figure*}

\begin{figure*}
\begin{center}
\scalebox{1.0}{\plotone{f2.eps}}
\end{center}
\caption{$\theta^{2}$ distribution from VERITAS observations of the Coma cluster of galaxies. The points with error bars represent the ON-source data sample and the filled area is the estimated background due to cosmic rays. Each bin represents an annulus around the Coma cluster core position. The data were derived from the ring background model using a $0.2^{\circ}$ sampling radius.}
\label{fig:thetasq}
\end{figure*}

\begin{figure*}
\begin{center}
\scalebox{0.75}{\plotone{f3.eps}}
\end{center}
\caption{Distribution of significance for Figure \ref{fig:skymaps} and a sampling radius of $0.2^{\circ}$. The line is a Gaussian fit to the data, with mean close to zero and almost unit standard deviation.}
\label{fig:sigdist}
\end{figure*}

\begin{figure*}
\begin{center}
%\scalebox{1.0}{\plotone{f4.eps}}
\end{center}
\caption{Integral flux upper limits (this work, Table X) compared with simulated integrated spectra of VHE gamma-ray emission from the Coma cluster.}
\label{fig:spectrum}
\end{figure*}

\begin{figure*}
\begin{center}
\scalebox{1.0}{\plotone{f5.eps}}
\end{center}
\caption{CR-to-thermal pressure, $X_{\mathrm{CR}}=\expval{P_{\CR}}/ \expval{P_{\mathrm{th}}}$, as a function of radial distance from the center of the Coma cluster. Model predictions are shown by dashed lines \citep{article:PinzkePfrommer:2010} while these are scaled to match the most constraining VERITAS upper limits within 0.2$^\circ$ (solid). We compare differential $X_\CR$-profiles (red) to integrated profiles $X_{\CR}(<R/R_{\rmn{vir}})$ (blue) which we use to compare to the upper limits.}
\label{fig:xcr}
\end{figure*}

\begin{figure*}
\begin{center}
\scalebox{1.0}{\plotone{f6.eps}}
\end{center}
\caption{Limits on the DM annihilation cross section $\expval{\sigma v}$ as a function of the DM particle mass $m_{\chi}$ derived from the VERITAS gamma-ray flux upper limits presented in this work.}
\label{fig:dm}
\end{figure*}

\begin{deluxetable}{lcc}
\tablecolumns{3}
\tablewidth{0pc}
\tablecaption{Regions of interest in the Coma cluster field of view}
\tablehead{
\colhead{Source} &
\colhead{RA} &
\colhead{Dec}
}
\startdata
Core & \RA{12}{59}{48.7} & \Dec{+27}{58}{50.0}\\
NGC 4889 & \RA{13}{00}{08.13} & \Dec{+27}{58}{37.03}\\
NGC 4874 & \RA{12}{59}{35.71} & \Dec{+27}{57}{33.37}\\
NGC 4921 & \RA{13}{01}{26.12} & \Dec{+27}{53}{09.59}\\
\enddata
\tablecomments{The cluster core is considered both as a point source and a modestly extended source. Three central galaxies are also considered in point-source searches. The choice is based on evidence of an excess of non-thermal X-ray emission \citep{article:Neumann_etal:2003} at the location of these galaxies.}
\label{table:roi}
\end{deluxetable}

\begin{deluxetable}{lccccccccc}
\tablecolumns{10}
\tablewidth{0pc}
\tablecaption{VERITAS flux upper limits for different regions of interest in the Coma cluster of galaxies and surroundings}
\rotate
\tablehead{
\colhead{Source} & \colhead{RoI\tablenotemark{a} [deg]} & \colhead{$N_{S}$} & \colhead{S\tablenotemark{b} [$\sigma$]} &
\multicolumn{6}{c}{Flux U.L.\tablenotemark{c}} \\
\colhead{} & \colhead{} & \colhead{} & \colhead{} & \multicolumn{2}{c}{$\alpha=2.1$} &
\multicolumn{2}{c}{$\alpha=2.3$} &
\multicolumn{2}{c}{$\alpha=2.5$}
}
\startdata
Core & 0 & 17 & 0.84 & 2.59 & (0.78\%) & 2.78 & (0.83\%) & 2.97 & (0.89\%) \\
& 0.2 & -41 & -1.0 & 1.96 & (0.59\%)  & 2.09 & (0.63\%) & 2.21 & (0.66\%) \\
& 0.4 & -26 & -0.30 & 4.44 & (1.3\%)  & 4.74 & (1.4\%) & 5.02 & (1.5\%)\\
NGC 4889 & 0 & 3 & 0.14 & - & - & 1.85 & (0.55\%)  & - & - \\
NGC 4874 & 0 & -14 & -0.71 & - & - &  1.51 & (0.45\%)  & - & - \\
NGC 4921 & 0 & -4 & -0.23 & - & - &  2.41 & (0.72\%)  & - & -
\enddata
\tablenotetext{a}{Intrinsic source radius (zero means point source), which is convolved with the analysis PSF.}
\tablenotetext{b}{Statistical significance calculated according to \citet{article:LiMa:1983}.}
\tablenotetext{c}{99\% confidence level upper limit in units of $10^{-8}$ ph. m$^{-2}$ s$^{-1}$ calculated according to \citet{article:Rolke_etal:2005} above an energy threshold of 220 GeV, with corresponding fluxes in percent of the steady Crab Nebula flux in parenthesis.}
\label{table:results}
\end{deluxetable}

\begin{deluxetable}{cccccccccccc}
\tablecolumns{12}
\tablewidth{0pc}
\tablecaption{Constraints on relative CR pressure in the Coma cluster core for different spatial extensions} \rotate
\tabletypesize{\small}
\tablehead{
\colhead{RoI\tablenotemark{a} [deg]} 
    & \colhead{$F_\rmn{sim}(E)$\tablenotemark{b}} & \colhead{$F_\rmn{sim}(>E)$\tablenotemark{c}} 
    & \multicolumn{3}{c}{analytic model: $X_{\CR,\,\rmn{max}}~[\%]$\tablenotemark{d} } 
    & \multicolumn{3}{c}{$F_{\gamma,\,\rmn{min}}(>E)$\tablenotemark{e}} 
    & \multicolumn{3}{c}{$X_{\CR,\,\rmn{min}}~[\%]$\tablenotemark{f} } \\
& & & \colhead{$\alpha=2.1$} & \colhead{$\alpha=2.3$} & \colhead{$\alpha=2.5$}
    & \colhead{$\alpha=2.1$} & \colhead{$\alpha=2.3$} & \colhead{$\alpha=2.5$}
    & \colhead{$\alpha=2.1$} & \colhead{$\alpha=2.3$} & \colhead{$\alpha=2.5$}
}
\startdata
0 & -     & -    & 10   & 23 & 97 & 1.6 & 0.7 & 0.3 & 0.067 & 0.061 & 0.11 \\
0.2  & 0.80  & 2.9  & 4.8  & 10 & 43 & 3.1 & 1.4 & 0.6 & 0.078 & 0.072 & 0.13 \\
0.4  & 1.21  & 4.4  & 6.7  & 15 & 62 & 6.3 & 2.8 & 1.3 & 0.098 & 0.090 & 0.16
\enddata
\tablenotetext{a}{Intrinsic source radius (zero means point source), which is convolved with the analysis PSF.}
\tablenotetext{b}{Differential gamma-ray flux $E^{2}dF/dE$ at $E=220$ GeV in units of $10^{-6}$ GeV m$^{-2}$ s$^{-1}$ in the simulation-based analytic model by \citet{article:PinzkePfrommer:2010}. }
\tablenotetext{c}{Integrated gamma-ray flux above $E=220~\rmn{GeV}$ in units of $10^{-9}$ ph. m$^{-2}$ s$^{-1}$; simulation-based analytical model by \citet{article:PinzkePfrommer:2010}.} 
\tablenotetext{d}{Constraint on the relative CR pressure, $X_{\CR} = \expval{P_\CR}/ \expval{P_\rmn{th}}$, which was assumed to be constant through the cluster and calculated according to \citet{article:PfrommerEnsslin:2004b}.}
\tablenotetext{e}{Minimum gamma-ray flux in units of $10^{-10}$ ph. m$^{-2}$ s$^{-1}$ of the hadronic model described in the text.}
\tablenotetext{f}{Minimum relative CR pressure, $X_{\CR,\,\rmn{min}} = F_{\rmn{min}}/F$, for energies $E>220~\rmn{GeV}$ in the hadronic model described in the text.}
\label{table:constraints}
\end{deluxetable}

\begin{deluxetable}{cccc}
\tablecolumns{4}
\tablewidth{0pc}
\tablecaption{Constraints on magnetic fields in the hadronic model of the radio halo of Coma cluster}
\tablehead{
\colhead{$\alpha_{B}$} &
\multicolumn{3}{c}{Minimum magnetic field, $B_0~[\mu\rmn{G}]$}\\
\colhead{} & \colhead{$\alpha=2.1$} & \colhead{$\alpha=2.3$} & \colhead{$\alpha=2.5$}
}
\startdata
0.3 & 0.61 & 0.51 & 0.43 \\
0.5 & 0.82 & 0.68 & 0.58 \\
0.7 & 1.19 & 0.98 & 0.83 
\enddata
\tablecomments{The parameters of the magnetic field are the magnetic decline, $\alpha_B$, and the central field strength, $B_{0}$, which are defined by $B(r) = B_{0}\,[n_\rmn{e}(r)/n_{\rmn{e}}(0)]^{\alpha_B}$.}
\label{table:Bmin}
\end{deluxetable}

\begin{deluxetable}{lcccccc}
\tablecolumns{7}
\tablewidth{0pc}
\tablecaption{\emph{Fermi}-LAT upper limits for the Coma cluster core}
\tablehead{
\colhead{Spatial model} & \multicolumn{3}{c}{S[$\sigma$]\tablenotemark{a}} & \multicolumn{3}{c}{Flux U.L.\tablenotemark{b}} \\
\colhead{} & \colhead{1-3 GeV} & \colhead{3-10 GeV} & \colhead{10-30 GeV} & 
\colhead{1-3 GeV} & \colhead{3-10 GeV} & \colhead{10-30 GeV} \\
\hline
\multicolumn{7}{c}{Spectral index $\alpha=2.1$ (fixed)}
}
\startdata
Point source & 0.30 & 0 & 1.3 & 1.73 & 4.87 & 0.627 \\
Disk, $r=0.2^{\circ}$ & 0.65 & 0.22 & 2.0 & 2.03 & 0.691 & 0.783 \\
Disk, $r=0.4^{\circ}$ & 0.77 & 0.18 & 2.3 & 2.39 & 0.757 & 0.847 \\
\cutinhead{Spectral index $\alpha=2.5$ (fixed)}
Point source & 0.33 & 0 & 1.42 & 1.81 & 0.526 & 0.635 \\
Disk, $r=0.2^{\circ}$ & 0.66 & 0.35 & 2.1 & 2.12 & 0.748 & 0.785 \\
Disk, $r=0.4^{\circ}$ & 0.80 & 0.36 & 2.4 & 2.50 & 0.833 & 0.849 \\
\enddata
\tablenotetext{a}{Statistical significance.}
\tablenotetext{b}{95\% confidence level upper limit in units of $10^{-10}$ ph. cm$^{-2}$ s$^{-1}$.}
\label{table:fermi}
\end{deluxetable}

\begin{deluxetable}{ccc}
\tablecolumns{3}
\tablewidth{0pc}
\tablecaption{Astrophysical factors}
\tablehead{
\colhead{RoI [deg]} & $\expval{J}_{\rmn{signal}}$ [GeV$^{2}$ cm$^{-5}$ sr] & $\alpha\expval{J}_{\rmn{bkg}}$ [GeV$^{2}$ cm$^{-5}$ sr]
}
\startdata
0 & $5.7\times 10^{16}$ & $1.3\times 10^{14}$ (negligible) \\
0.2 & $8.1\times 10^{16}$ & $4.4\times 10^{14}$ ($<0.01\expval{J}_{\rmn{signal}}$, negligible) \\
0.4 & $9.4\times 10^{16}$ & $1.3\times 10^{15}$ ($\simeq0.01\expval{J}_{\rmn{signal}}$, negligible)
\enddata
\tablecomments{$\expval{J}_{\rmn{bkg}}$ is the astrophysical factor calculated for the background region (ring method) and is used to estimate the level of gamma-ray contamination from DM.}
\end{deluxetable}

\begin{deluxetable}{lccc}
\tablecolumns{4}
\tablewidth{0pc}
\tablecaption{Limits on the DM annihilation cross section $\expval{\sigma v}$.}
\tablehead{
\colhead{Channel} & RoI [deg] & $m_{\chi}$ [GeV] & $\expval{\sigma v}$ [cm$^{3}$ s$^{-1}$]
}
\startdata
W$^{+}$W$^{-}$ & 0 & 2000 & $1.1\times 10^{-20}$ \\
& 0.2 & 1900 & $4.3\times 10^{-21}$ \\
& 0.4 & 1900 & $8.4\times 10^{-21}$ \\
$b\bar{b}$ & 0 & 3500 & $1.2\times 10^{-20}$ \\
& 0.2 & 3400 & $4.4\times 10^{-21}$ \\
& 0.4 & 3500 & $8.7\times 10^{-21}$ \\
$\tau^{+}\tau^{-}$ & 0 & 670 & $2.4\times 10^{-21}$ \\
& 0.2 & 650 & $9.1\times 10^{-22}$ \\
& 0.4 & 660 & $1.8\times 10^{-21}$ \\
\enddata
\tablecomments{Limits are for different integration regions and DM particle mass $m_{\chi}$ derived from the VERITAS gamma-ray flux upper limits presented in this work.}
\end{deluxetable}

\end{document}

