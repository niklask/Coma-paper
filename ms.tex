\documentclass[12pt,manuscript]{aastex}
%%\documentclass[twocolumn]{emulateapj}

\usepackage{lscape}

\newcommand{\vdag}{(v)^\dagger}
\newcommand{\myemail}{nkarlsson@physics.umn.edu}
\newcommand{\expval}[1]{\left\langle #1 \right\rangle}
\newcommand{\RA}[3]{#1$^{\mathrm{h}}$#2$^{\mathrm{m}}$#3$^{\mathrm{s}}$}
\newcommand{\Dec}[3]{#1$^{\circ}$#2\arcmin#3\arcsec}

\shorttitle{Coma Cluster Observations with VERITAS}
\shortauthors{VERITAS collaboration}

\begin{document}

%\title{VHE Gamma-Ray Observations of the Coma Cluster of Galaxies with VERITAS}
\title{Constraints on Cosmic-Ray Particle Populations, Magnetic Fields, and Dark Matter from VHE Gamma-Ray Observations of the Coma Cluster of Galaxies with VERITAS}
\author{N. Karlsson\altaffilmark{1,2}, M. Pohl\altaffilmark{3,4}}
\affil{(the VERITAS collaboration)} 
%%\email{nkarlsson@physics.umn.edu}
\and
\author{C. Pfrommer\altaffilmark{5,6},
A. Pinzke\altaffilmark{7}}

\altaffiltext{1}{Corresponding author: N. Karlsson, nkarlsson@physics.umn.edu}
\altaffiltext{2}{School of Physics and Astronomy, University of Minnesota, Minneapolis, MN 55455, USA}
\altaffiltext{3}{DESY, Platanenallee 6, 15738 Zeuthen, Germany}
\altaffiltext{4}{ Institut fu\"r Physik und Astronomie, Universit\"at Potsdam, 14476 Potsdam-Golm, Germany}
\altaffiltext{5}{Heidelberg Institute	for Theoretical	Studies, Schloss-Wolfsbrunnenweg 35, D-69118 Heidelberg, Germany}
\altaffiltext{6}{Canadian Institute for Theoretical Astrophysics, University of Toronto, Toronto, ON, M5S 3H8, Canada}
\altaffiltext{7}{University of California, Santa Barbara, CA 93106}

%%\journalinfo{The Astrophysical Journal}
%%\submitted{preprint}

\begin{abstract}
Theoretical models of galaxy clusters suggest they may be sources of very high-energy (VHE) gamma-ray emission. Observations of radio halos and relics in clusters indicate populations of relativistic electrons and positrons, which are either injected into the intra-cluster medium and accelerated by turbulent shocks or secondaries from interactions of cosmic-ray hadrons (protons and ions) with the intra-cluster gas. Those interactions would also produce gamma rays through the decay of neutral pions. Reported here are observations of the Coma cluster of galaxies with the VERITAS array of imaging Cherenkov telescopes. The observations were made between March and May 2008. Radio observations at 1.41 GHz taken with the Green Bank Telescope and X-ray observations with the ROSAT All-Sky survey was used to a priori select regions of interest in the field of view. No significant VHE gamma-ray emission from the Coma cluster was detected by VERITAS. Flux upper limits at the 99\% confidence level were obtained for the cluster core, both as a point source and a mildly extended source centered at the cluster core, and for a select few other cluster members. The gamma-ray flux upper limits have been used to constrain the populations of cosmic-ray particles and magnetic fields in the cluster. The cosmic-ray pressure is... The observations imply a lower limit on the intra-cluster magnetic field of X$\mu$G. Finally, a constraint on the dark matter annihilation cross section is derived, $\expval{\sigma v}=$.
\end{abstract}

%\keywords{cosmic rays: interactions --- galaxies --- galaxy clusters -- VHE}
\keywords{cosmic rays --- gamma rays: observations --- gamma rays: VHE --- galaxies: clusters: general --- galaxies: clusters: individual: Coma (ACO 1656)}

\section{Introduction}
Clusters of galaxies are the largest observable structures in the Universe, with typical diameters of several Mpc and masses on the order of $10^{14}$ to $10^{15} M_{\odot}$. These gravitationally bound objects have formed on increasingly larger scales during the evolution of the Universe through the process of cosmic structure formation \citep{article:Voit:2005}. Although most of the cluster mass comes from large amounts of dark matter embedded in a cluster, as indicated by cluster dynamics and weak gravitational lensing \citep[][]{article:DiaferioSchindlerDolag:2008}, gas making up the intra-cluster medium (ICM) contributes about 15\% of the total cluster mass. Individual galaxies contribute only a small amount (about 5\%) to the cluster mass. The ICM gas mass also comprise a significant fraction of the observable (baryonic) matter in the Universe.

The ICM is a hot, $T\sim 10^{8}$ K, plasma emitting thermal bremsstrahlung radiation in the soft X-rays regime \citep[see, e.g.][]{article:Petrosian:2001}. Heating of this plasma is thought to come primarily from collisionless structure-formation shocks, due to accretion and merger processes. Such shocks and turbulences in the ICM gas and magnetic fields also provide a means to efficiently accelerate particles \citep[see, e.g.,][]{article:ColafrancescoBlasi:1998, article:Ryu_etal:2003}. Many clusters do indeed feature mega-parsec scale halos of non-thermal radio emission, indicative of a population of relativistic electrons and positrons (leptons) and magnetic fields permeating the ICM \citep{article:Cassano_etal:2010}. Recent observations of possible non-thermal emission from clusters in the the extreme ultraviolet \citep[EUV; ][]{article:SarazinLieu:1998} and hard X-rays \citep{article:RephaeliGruber:2002, article:Fusco-Femiano_etal:2004, article:Eckert_etal:2007} provide further indications of relativistic particle populations in clusters, although the interpretation of these observations have been disputed \citep[see, e.g.][]{article:Wik_etal:2009}.

Galaxy clusters have for many years been proposed as sources of gamma rays. If shock acceleration in the ICM is an efficent process, a population of highly relativistic cosmic-ray (CR) hadrons, in the form of protons and heavy ions, in the ICM is to be expected. The main energy-loss mechanism for CR hadrons at high energies is pion production through the interaction of CRs with nuclei in the ICM. Pions are short lived and decay; the charged pions to electrons and positrons,\footnote{A charged pion decays to a muon or antimuon, which subsequently decay to an electron or positron.} and the neutral pion to gamma rays. Due to the low density of the ICM ($n_{\mathrm{ICM}}\sim 10^{-3}$ cm$^{-3}$) and the sheer size of a cluster, CR hadrons will be confined in the cluster on very long timescales \citep[][]{article:Volk_etal:1996, article:Berezinsky_etal:1997} allowing them to accumulate over time. 

Depending on the CR energy density, quantified by the fraction of energy in CRs to the thermal energy in a cluster, $\eta=\expval{ E_{\mathrm{CR}}/E_{\mathrm{th}}}$, an observable flux of high-energy gamma rays can be expected. This fraction could plausibly be as large as 50\% \citep{article:Ryu_etal:2003}, but a much smaller fraction of only a few percent is required in order to produce a gamma-ray flux observable with current generation gamma-ray telescopes. 

Gamma-ray emission can also be produced by Compton up-scattering of ambient photons, for example cosmic microwave background (CMB) photons, on ultra-relativistic leptons. Those leptons can either be secondaries from the above mentioned CR interactions or be injected into the ICM by powerful cluster members and be further accelerated by diffusive shock acceleration or turbulent reacceleration processes.

While several observations of clusters of galaxies have been made with satellite-borne and ground-based telescopes, a detection of gamma-ray emission from a cluster has yet to be made.  Observations with EGRET \citep{article:Sreekumar_etal:1996, article:Reimer_etal:2003} and the Fermi Gamma-ray Space Telescope \citep{article:Ackermann_etal:2010} have provided upper limits on the gamma-ray fluxes (typically $\sim10^{-9}$ ph cm$^{2}$ s$^{-1}$ for Fermi observations) for several galaxy clusters in the MeV to GeV band. Upper limits on the VHE gamma-ray flux from a few select clusters, including the Coma cluster, have been provided by observations with ground-based imaging atmospheric Cherenkov telescopes \citep[IACTs;][]{article:Perkins_etal:2006, inproc:Perkins_etal:2008, article:Aharonian_etal:2009, article:Aleksic_etal:2010}.

The Coma cluster of galaxies (ACO 1656) is one of the most thouroughly studied clusters over all wavelengths \citep{article:Voges_etal:1999}. Located at a distance of about 100 Mpc \citep[$z=0.023$;][]{article:StrubleRood:1999}, it is one of the closest and also one of the most massive clusters \citep[$M \sim 10^{15}M_{\odot}$;][]{article:Smith:1983, article:Kubo_etal:2008}. It hosts both a giant radio halo \citep{article:Giovannini_etal:1993,article:Thierbach_etal:2003} and peripheral radio relic, which appears connected to the radio halo with a ``diffuse'' bridge \citep[see discussion in][]{article:BrownRudnick:2010}. It has been suggested that the relic is an infall shock. Extended soft X-ray (SXR) emission is evident from the ROSAT all-sky survey in the 0.1 to 2.4 keV band \citep{article:BrielHenryBohringer:1992} and observations with XMM-Newton \citep{article:Briel_etal:2001} revealed substructure in the X-ray halo. As such it is a natural candidate for gamma-ray observations. In this article, results from the VERITAS observations of the Coma cluster of galaxies are reported, together with the resulting constraints on cosmic-ray particle populations, magnetic fields, and dark matter in the cluster.

\section{VERITAS Observations and Analysis}
The VERITAS gamma-ray detector \citep{article:Weekes_etal:2002} is an array of four 12 m diameter imaging atmospheric Cherenkov telescopes \citep[IACTs;][]{article:Holder_etal:2006} located at an altitude of 1268 m a.s.l. at the Fred Lawrence Whipple Observatory in southern Arizona (31$^{\circ}$~40\arcmin~30\arcsec~N, 110$^{\circ}$~57\arcmin~07\arcsec~W). Each of the telescopes is equipped with a 499 pixel camera covering a 3.5$^{\circ}$ field of view. The array, completed in the spring of 2007, is designed to detect gamma-ray emission from astrophysical objects in the energy range from 100 GeV to more than 30 TeV. The energy resolution is about 15\% and the angular resolution (68\% containment) is about 0.1$^{\circ}$ per event at 1 TeV. The sensitivity of the array allows for detection of a point source with a flux of 1\% of the steady Crab Nebula flux above 300 GeV at the $5\sigma$ level in under 30 hours.\footnote{The integral flux sensitivity above 300 GeV was improved by about 30\% with the move of one telescope in the summer of 2009. The move had no impact on the Coma cluster data set.}

The Coma cluster was observed with VERITAS between March and May in 2008 with all four telescopes fully operational. The total exposure amounts to 18.6 hours of quality-selected live time, i.e. time periods of astronomical darkness, clear sky conditions and without technical problems with the array. The center of the cluster was tracked in \emph{wobble} mode, where the expected source location is offset from the center of the field of view, to allow for simultaneous background estimation \citep{article:Formin_etal:1994}. All of the observations were made in a small range of zenith angles averaging $\sim 21^{\circ}$.

The data analysis was performed following the standard VERITAS procedures described in \citet{inproc:Cogan_etal:2007} and \citet{inproc:Daniel_etal:2007}. Prior to event reconstruction and selection, all shower images are calibrated and cleaned. Showers are then reconstructed for events with at least two contributing telescope images that pass the following quality selection criteria: more than four participating pixels in the camera, number of photoelectrons in the image is larger than 75, and the distance from the image centroid to the center of the camera is less than $1.43^{\circ}$. These quality selection criteria impose an energy threshold\footnote{Defined as$\ldots$} of about 220 GeV. In addition, events for which only images from the two closest-spaced telescopes\footnote{In the array configuration prior to summer 2009, two telescopes had a separation of only 35 m.} survive quality selection are rejected, as they introduce an irreducible high background rate, due to local muons, degrading the instrument sensitivity \citep{article:MaierKnapp:2007}.

Gamma-ray like events are separated from the CR background by imposing cuts on scaled parameters \citep{article:Aharonian_etal:1997, article:Krawczynski_etal:2006} calculated from a parameterized moment analysis of the shower images \citep{inproc:Hillas:1985}. The gamma-hadron separation cuts used in this analysis were optimized a priori for a weak point source (3\% Crab Nebula flux level) and a photon index of -2.4 using data taken on the Crab Nebula during the same epoch and are termed \emph{standard} cuts. Because the VHE gamma-ray spectrum for the Coma cluster is expected to be hard, photon index of about -2.3, these standard cuts are suitable for the analysis of the Coma cluster data set. 

The Coma cluster is a very rich cluster of galaxies with many plausible sites for gamma-ray emission. The core region, the peripheral radio relic, as well as individual powerful cluster member galaxies. VERITAS has a large enough field of view to allow investigation of several of these scenarios. In this work the focus has been on the core region and three cluster members. The latter are treated as point sources in the analysis, whereas the core is treated as both a point source and a mildly extended source. The intrinsic source extension is assumed to be $0.2^{\circ}$ and $0.4^{\circ}$, based on the extension of the thermal soft X-ray emission from the core. There is evidence of a recent merger event between the two central galaxies NGC 4889 and NGC 4874 \citep{article:Tribble:1993}. There is also evidence for an excess of non-thermal X-ray emission from these galaxies as well as the galaxy NGC 4921. Therefore, searches for point-like VHE gamma-ray emission have been conducted at the locations of these galaxies. The regions of interest considered in this work are summarized in Table \ref{table:roi}.

The ring background model \citep{article:Aharonian_etal:2001} is used to estimate the background due to CRs. The total number of events in a given region of interest is then compared to the estimated background from the off-source region scaled by the ratio of the solid angles to produce a final excess. The VHE gamma-ray significance is then calculated using the method by \citet{article:LiMa:1983}.  Significance skymaps over the VERITAS field of view produced with a $0.2^{\circ}$ sampling radius are shown in Figure \ref{fig:skymaps} with overlaid X-ray and radio contours from the ROSAT all-sky survey \citep{article:BrielHenryBohringer:1992}  and GBT 1.4 GHz observations \citep{article:BrownRudnick:2010} respectively.

\section{Fermi-LAT Observations and Analysis}

\section{Results}
No significant excess of VHE gamma-rays from the Coma cluster was detected with VERITAS, as illustrated by the $\theta^{2}$ distribution shown in  Figure \ref{fig:thetasq}. The $\theta^{2}$ distribution is a radial comparison of the ON-source event distribution with the estimated background. It extends out to a radius of $0.42^{\circ}$ to cover all regions of interest considered in this work. A 99\% confidence level upper limit is calculated for each region of interest using the method described by \citet{article:Rolke_etal:2005} assuming a Gaussian background. A lower bound of zero is imposed on the signal rate to handle the negative excesses seen for several of the regions of interest. Figure \ref{fig:sigdist} shows that the VERITAS data is well fit by a Gaussian with a mean close to zero and a standard deviation within a few percent of unity. 

Table \ref{table:results} lists the upper limits for the selected regions of interest tabulated in Table \ref{table:roi}. These upper-limit calculations depend on the gamma-ray spectrum, which in this work is assumed to be a power law with a spectral index $\alpha=-2.1, -2.3$, and $-2.5$. A power-law spectrum with index $\alpha=-1.5$ is also considered for the dark matter constraints.

\section{Cosmic Rays}

\section{Constrainsts on Magnetic Fields}

\section{Dark Matter}

\section{Discussion}
VERITAS observed the Coma cluster of galaxies for a total of 18.6 hours of high-quality live time between March and May in 2008. 

\acknowledgments
This research was supported by grants from the U.S. Department of Energy, the U.S. National Science Foundation, and the Smithsonian Institution, by NSERC in Canada, by Science Foundation Ireland, and by STFC in the UK. We acknowledge the excellent work of the technical support staff at the FLWO and the collaborating institutions in the construction and operation of the instrument.

%%Facilities: \facility{VERITAS}.

\bibliographystyle{apj}
\bibliography{refs}

\begin{figure*}
\begin{center}
\scalebox{0.45}{\plotone{f1a.eps}}
\scalebox{0.45}{\plotone{f1b.eps}}
\end{center}
\caption{\emph{Left}: Smoothed significance map of the Coma cluster calculated from the observed excess VHE gamma-ray events over a $4.5^{\circ}\times 4.5^{\circ}$ field of view. The excess counts were derived using a ring background model \citep{article:Aharonian_etal:2001}. White contours show the X-ray counts per second in the 0.1 to 2.4 keV energy band from the ROSAT all-sky survey \citep{article:BrielHenryBohringer:1992}. \emph{Right}: Same as above but with overlaid contours from point source subtracted GBT radio observations at 1.4 GHz \citep{article:BrownRudnick:2010}. Shown are also the $0.2^{\circ}$ and $0.4^{\circ}$ radii (dashed cyan) considered for the extended source analyses presented here.}
\label{fig:skymaps}
\end{figure*}

\begin{figure*}
\begin{center}
\scalebox{1.0}{\plotone{f2.eps}}
\end{center}
\caption{$\theta^{2}$ distribution from VERITAS observations of the Coma cluster of galaxies. The points with error bars represent the ON-source data sample and and the filled area the background sample. Each bin represents an annulus around the Coma cluster core position. The data were derived from the ring background model using a $0.2^{\circ}$ sampling radius.}
\label{fig:thetasq}
\end{figure*}

\begin{figure*}
\begin{center}
\scalebox{0.75}{\plotone{f3.eps}}
\end{center}
\caption{Distribution of significance for Figure \ref{fig:skymaps} and a sampling radius of $0.2^{\circ}$. The line is a Gaussian fit to the data, with mean close to zero and almost unit standard deviation.}
\label{fig:sigdist}
\end{figure*}

\begin{figure*}
\begin{center}
%\scalebox{1.0}{\plotone{f4.eps}}
\end{center}
\caption{Integral flux upper limits (this work, Table X) compared with simulated integrated spectra of VHE gamma-ray emission from the Coma cluster.}
\label{fig:spectrum}
\end{figure*}

\begin{figure*}
\begin{center}
\scalebox{1.0}{\plotone{f5.eps}}
\end{center}
\caption{Cosmic-ray pressure, $X_{\mathrm{CR}}$, as a function of radial distance from the Coma cluster core.}
\label{fig:xcr}
\end{figure*}

\begin{figure*}
\begin{center}
\scalebox{1.0}{\plotone{f6.eps}}
\end{center}
\caption{Limits on the DM annihilation cross section $\expval{\sigma v}$ calculated using the VERITAS data in this work.}
\label{fig:dm}
\end{figure*}

\begin{deluxetable}{lcc}
\tablecolumns{3}
\tablewidth{0pc}
\tablecaption{Regions of interest in the Coma cluster field of view. The cluster core is considered both as a point source and a modestly extended source. Three central galaxies are also considered in point-source searches. The choice is based on evidence of an excess of non-thermal X-ray emission \citep{article:Neumann_etal:2003} at the location of these galaxies.}
\tablehead{
\colhead{Source} &
\colhead{RA} &
\colhead{Dec}}
\startdata
Core & \RA{12}{59}{48.7} & \Dec{+27}{58}{50.0}\\
NGC 4889 & \RA{13}{00}{08.13} & \Dec{+27}{58}{37.03}\\
NGC 4874 & \RA{12}{59}{35.71} & \Dec{+27}{57}{33.37}\\
NGC 4921 & \RA{13}{01}{26.12} & \Dec{+27}{53}{09.59}\\
\enddata
\label{table:roi}
\end{deluxetable}

\begin{landscape}
\begin{deluxetable}{lcccccccccccc}
\tablecolumns{6}
\tablewidth{0pc}
\tablecaption{Flux upper limits for regions of interest in the Coma cluster of galaxies and surroundings.}
\tablehead{
\colhead{Source} & \colhead{RoI\tablenotemark{a} [deg]} & \colhead{$N_{S}$} & \colhead{S\tablenotemark{b} [$\sigma$]} &
\multicolumn{6}{c}{Flux U.L.\tablenotemark{c}} &
\multicolumn{3}{c}{Minimum gamma-ray flux\tablenotemark{d}}\\
\colhead{} & \colhead{} & \colhead{} & \colhead{} & \multicolumn{2}{c}{$\alpha=-2.1$} &
\multicolumn{2}{c}{$\alpha=-2.3$} &
\multicolumn{2}{c}{$\alpha=-2.5$} &
\colhead{$\alpha=-2.1$} &
\colhead{$\alpha=-2.3$} &
\colhead{$\alpha=-2.5$}}
\startdata
Core & 0 & 17 & 0.84 & 2.59 & (0.78\%) & 2.78 & (0.83\%) & 2.97 & (0.89\%) \\
& 0.2 & -41 & -1.0 & 1.96 & (0.59\%)  & 2.09 & (0.63\%) & 2.21 & (0.66\%) \\
& 0.4 & -26 & -0.30 & 4.44 & (1.3\%)  & 4.74 & (1.4\%) & 5.02 & (1.5\%)\\
NGC 4889 & 0 & 3 & 0.14 & - & - & 1.85 & (0.55\%)  & - & - \\
NGC 4874 & 0 & -14 & -0.71 & - & - &  1.51 & (0.45\%)  & - & - \\
NGC 4921 & 0 & -4 & -0.23 & - & - &  2.41 & (0.72\%)  & - & -
\enddata
\tablenotetext{a}{Intrinsic source radius (zero means point source), which is convolved with the analysis PSF.}
\tablenotetext{b}{Statistical significance calculated according to \citet{article:LiMa:1983}.}
\tablenotetext{c}{99\% confidence level upper limit in units of $10^{-8}$ ph. m$^{-2}$ s$^{-1}$ calculated according to \citet{article:Rolke_etal:2005} above an energy threshold of 220 GeV. Values in parentheses are the corresponding fluxes in percent of the steady Crab Nebula flux.}
\tablenotetext{d}{Minimum gamma-ray flux in units of $10^{-13}$ ph. cm$^{-2}$ s$^{-1}$ from the hadronic model described in the text.}
\label{table:results}
\end{deluxetable}
\end{landscape}

\end{document}

